%
% Header die benutzt werden sollen
%
\documentclass[
  % pdftex,
  fontsize=12pt,          % Schriftgroesse
  DIV10,                  % Angabe bzgl Bestimmung der Seitenabstaende
  ngerman,                % fuer Umlaute, Silbentrennung etc.
  paper=a4,               % Papierformat
  twoside=false,          % einseitiges Dokument
  titlepage,              % es wird eine Titelseite verwendet
  parskip=half,           % Abstand zwischen Absaetzen (halbe Zeile)
  headings=normal,        % Groesse der Ueberschriften verkleinern
  %liststotoc, % deprecated lt. LOG !!!
  listof=nochaptergap,  % Verzeichnisse im Inhaltsverzeichnis auffuehren.
  % bibtotoc,   % deprecated lt. LOG !!!
  bibliography=totoc, % Literaturverzeichnis im Inhaltsverzeichnis auffuehren
  index=totoc,            % Index im Inhaltsverzeichnis auffuehren
  captions=tableheading,  % Beschriftung von Tabellen oberhalb ausgeben
%  bookmarksopen,
%  bookmarksnumbered,
  final                 % Status des Dokuments (final/draft)
]{scrreprt}
\usepackage{scrhack}

% Zeilenabstand
\usepackage[onehalfspacing]{setspace}
%\usepackage{setspace}
%\setstretch{1,5}

% Code
\usepackage{listings}
\lstset{
	captionpos=b,
	literate=
		{á}{{\'a}}1 {é}{{\'e}}1 {í}{{\'i}}1 {ó}{{\'o}}1 {ú}{{\'u}}1
		{Á}{{\'A}}1 {É}{{\'E}}1 {Í}{{\'I}}1 {Ó}{{\'O}}1 {Ú}{{\'U}}1
		{à}{{\`a}}1 {è}{{\`e}}1 {ì}{{\`i}}1 {ò}{{\`o}}1 {ù}{{\`u}}1
		{À}{{\`A}}1 {È}{{\'E}}1 {Ì}{{\`I}}1 {Ò}{{\`O}}1 {Ù}{{\`U}}1
		{ä}{{\"a}}1 {ë}{{\"e}}1 {ï}{{\"i}}1 {ö}{{\"o}}1 {ü}{{\"u}}1
		{Ä}{{\"A}}1 {Ë}{{\"E}}1 {Ï}{{\"I}}1 {Ö}{{\"O}}1 {Ü}{{\"U}}1
		{â}{{\^a}}1 {ê}{{\^e}}1 {î}{{\^i}}1 {ô}{{\^o}}1 {û}{{\^u}}1
		{Â}{{\^A}}1 {Ê}{{\^E}}1 {Î}{{\^I}}1 {Ô}{{\^O}}1 {Û}{{\^U}}1
		{œ}{{\oe}}1 {Œ}{{\OE}}1 {æ}{{\ae}}1 {Æ}{{\AE}}1 {ß}{{\ss}}1
		{ű}{{\H{u}}}1 {Ű}{{\H{U}}}1 {ő}{{\H{o}}}1 {Ő}{{\H{O}}}1
		{ç}{{\c c}}1 {Ç}{{\c C}}1 {ø}{{\o}}1 {å}{{\r a}}1 {Å}{{\r A}}1
		{€}{{\euro}}1 {£}{{\pounds}}1 {«}{{\guillemotleft}}1
		{»}{{\guillemotright}}1 {ñ}{{\~n}}1 {Ñ}{{\~N}}1 {¿}{{?`}}1,
	numbers=left,
}

%Tabellen
\usepackage{array}
\usepackage{tabularx}
\usepackage{supertabular}
\usepackage{longtable}
\usepackage{hhline}
\usepackage{multirow}

%Grafiken
\usepackage{graphicx}
\usepackage{wrapfig}
\usepackage[svgnames, table]{xcolor}
\usepackage{floatflt}
\usepackage{float}

%Sprache und Anführungszeichen
\usepackage[T1]{fontenc}
\usepackage[utf8]{inputenc}
\usepackage[ngerman]{babel}
\usepackage{hyphenat}
\usepackage{courier}
\usepackage{palatino}
\usepackage[babel,german=quotes]{csquotes}
\usepackage{eurosym}

%Abstände
\usepackage[
	margin=25mm,
 	includefoot,
%	showframe=true
]{geometry}

% Pakete um Textteile drehen zu können, oder eine Seite Querformat anzeigen kann.
\usepackage{rotating}
\usepackage{lscape}
\usepackage{pdflscape}
\usepackage{enumerate}

%Literaturverweise
\usepackage[
 	style=libstyle,
	backend=biber
]{biblatex}
\renewcommand*{\multinamedelim}{\slash\space}
\renewcommand*{\finalnamedelim}{\multinamedelim}
\renewcommand*{\labelnamepunct}{\addcolon\newline}
\renewcommand*{\newunitpunct}{\addcomma\space}
\renewcommand*{\postnotedelim}{\addcomma\space S.\space}
\renewcommand*{\bibfootnotewrapper}{}
\renewcommand*{\finentrypunct}{}
\DefineBibliographyStrings{ngerman}{
	ibidem = {ebenda},
	urlseen = {Einsichtnahme:},
	edition = {Auflage}
}
%\usepackage[round]{natbib}
%\usepackage{cite}
\setcounter{biburllcpenalty}{7000}
\setcounter{biburlucpenalty}{8000}

% Hurenkinder und Schusterjungen verhindern
% http://projekte.dante.de/DanteFAQ/Silbentrennung
\clubpenalty=10000
\widowpenalty=10000
\displaywidowpenalty=10000

%Abkürzungsverzeichnis
\usepackage[printonlyused]{acronym}

% Notizen und To-dos
\usepackage[
%	disable,		% Notizen ausblenden
ngerman,			% Deutsche Listennamen
colorinlistoftodos,	% Farbe in Verzeichnis einblenden
]{todonotes}

% Fussnoten
\usepackage[hang, multiple, stable]{footmisc}
\usepackage{chngcntr}
\counterwithout{figure}{chapter} 
\counterwithout{equation}{chapter}
\counterwithout{footnote}{chapter}
\counterwithout{table}{chapter}
\AtBeginDocument{\counterwithout{lstlisting}{chapter}}

\usepackage{hyperref}
\usepackage[all]{hypcap}

%\usepackage[nottoc]{tocbibind}

\usepackage{tocloft}
\tocloftpagestyle{scrheadings}

\makeatletter
\begingroup\let\newcounter\@gobble\let\setcounter\@gobbletwo
  \globaldefs\@ne \let\c@loldepth\@ne
  \newlistof{listings}{lol}{\lstlistlistingname}
\endgroup
\let\l@lstlisting\l@section
\makeatother

\usepackage[
  automark,     % Kapitelangaben in Kopfzeile automatisch erstellen
  headsepline,  % Trennlinie unter Kopfzeile
  ilines        % Trennlinie linksbndig ausrichten
]{scrpage2}

% Kopf- und Fusszeilen ------------------------------------------------------
\pagestyle{scrheadings}

% Kopf- und Fusszeile auch auf Kapitelanfangsseiten -------------------------
\renewcommand*{\chapterpagestyle}{scrheadings}

% Schriftform der Kopfzeile -------------------------------------------------
\renewcommand{\headfont}{\normalfont}

% Kopfzeile -----------------------------------------------------------------
\ihead{\headmark} %\small{\untertitel}
\chead{\hspace{1mm}}
\ohead{Seite \thepage}
\setlength{\headheight}{21mm} % Höhe der Kopfzeile
\setheadsepline[text]{0.4pt} % Trennlinie unter Kopfzeile

% Fusszeile -----------------------------------------------------------------
%\ifoot{\copyright\ Samuel Philipp}
\cfoot{\hspace{1mm}}
%\ofoot{Seite \thepage}
% \setlength{\footheight}{21mm} % Höhe der Fußzeile
%\setfootsepline[text]{0.4pt} % Trennlinie über Fußzeile

%\automark[section]{chapter}
\usepackage{soul}
\newcommand{\hlc}[2][yellow]{{\sethlcolor{#1}\hl{#2}}}

% Markierungsfarben
\newcommand{\daniel}[1]{\hlc[cyan]{#1}}
\newcommand{\jani}[1]{\hlc[green]{#1}}
\newcommand{\samuel}[1]{\hlc[yellow]{#1}}

% \usepackage{ifthen}
% \renewcommand{\footcite}[2][]{\footnote{Vgl. \citetitle{#2}\ifthenelse{\equal{#1}{}}{}{ #1} \cite{#2}}}

\newcommand{\appref}[1]{\hyperref[app:#1]{\MakeUppercase{#1}}}

\lstdefinelanguage{JavaScript}{
  morekeywords={typeof, new, true, false, catch, function, return, null, catch, switch, var, if, in, while, do, else, case, break},
  morecomment=[s]{/*}{*/},
  morecomment=[l]//,
  morestring=[b]",
  morestring=[b]'
}

\definecolor{eclipsejavakeywords}{HTML}{7F0055}
\definecolor{eclipsejavacomments}{HTML}{3F7F5F}
\definecolor{eclipsejavastrings}{HTML}{0000FF}
\definecolor{eclipsejavaannotations}{HTML}{646464}
\definecolor{eclipsejavaidentifiers}{HTML}{0000C0}

\definecolor{eclipsexmlbasic}{HTML}{000000}
\definecolor{eclipsexmlcomments}{HTML}{800000}
\definecolor{eclipsexmlstrings}{HTML}{008000}

\definecolor{eclipsehtmlkeywords}{HTML}{7F0055}
\definecolor{eclipsehtmlcomments}{HTML}{3F7F5F}
\definecolor{eclipsehtmlstrings}{HTML}{0000FF}
\definecolor{eclipsehtmlbasic}{HTML}{000000}

\definecolor{link}{HTML}{0000FF}

\lstdefinestyle{eclipse}{
	float,
	frame=single,
	basicstyle=\ttfamily,
	showstringspaces=false,
	showspaces=false,
	numbers=left,
	captionpos=b,
	belowcaptionskip=4pt,
	breaklines=true
}

\lstdefinestyle{eclipsejava}{
	language=Java,
	style=eclipse,
	keywordstyle=\bfseries\color{eclipsejavakeywords},
	commentstyle=\color{eclipsejavacomments},
	stringstyle=\color{eclipsejavastrings},
	moredelim=[is][\textcolor{eclipsejavaannotations}]{\$\$}{\$\$},% für annotations
	moredelim=[is][\textcolor{eclipsejavaidentifiers}]{\#}{\#},% für variablennamen
	moredelim=[is][\itshape\textcolor{eclipsejavaidentifiers}]{\#\#}{\#\#},% für konstanten
	morekeywords={enum}
}

\lstdefinestyle{eclipsejavascript}{
	language=JavaScript,
	style=eclipse,
	keywordstyle=\bfseries\color{eclipsejavakeywords},
	commentstyle=\color{eclipsejavacomments},
	stringstyle=\color{eclipsejavastrings}
}

\lstdefinestyle{eclipsexml}{
	language=XML,
	style=eclipse,
	commentstyle=\color{eclipsexmlcomments},
	keywordstyle=\color{eclipsexmlbasic},
	identifierstyle=\color{eclipsexmlbasic},
	stringstyle=\color{eclipsexmlstrings},
	moredelim=[s][\textcolor{eclipsexmlcomments}]{<!--}{-->}
}

\lstdefinestyle{eclipsehtml}{
	language=HTML,
	style=eclipse,
	commentstyle=\color{eclipsehtmlcomments},
	keywordstyle=\color{eclipsehtmlkeywords},
	identifierstyle=\color{eclipsehtmlbasic},
	stringstyle=\color{eclipsehtmlstrings}
}

\lstdefinestyle{eclipseproperties}{
	style=eclipse,
	commentstyle=\color{eclipsejavacomments},
	keywordstyle=\color{eclipsejavakeywords},
	morecomment=[l]{\#},
	moredelim=[il][\textcolor{eclipsejavastrings}]{\%}{}
}

\lstdefinestyle{nonumbers}{
	style=eclipse,
    numbers=none,
	belowcaptionskip=0pt,
	moredelim=[is][\color{link}\underbar]{(*}{*)}
}

%\newcommand{\todo}[0]{\textbf{\textit{\textcolor{red}{TODO }}}}
%\newcommand{\progress}[0]{\textbf{\textit{\textcolor{orange}{In Progress }}}}

% besserer Blocksatz
\sloppy

%
% EOF
%
