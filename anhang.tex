\addchap{Anhang}

\section*{TEIL A: Autoren der einzelnen Kapitel}
Auf den folgenden Seiten werden die Kapitel in den Farben der Autoren markiert.
Dabei steht die Farbe blau für \daniel{Daniel Brown}, grün für \jani{Jan-Eric Gaidusch} und gelb für \samuel{Samuel Philipp}.

\rule{\textwidth}{1pt}

\daniel{Abstract}

{1 Einleitung}

\samuel{1.1 Einf\"uhrung}

\jani{1.2 Hintergrund}

\samuel{1.3 Aufgabenstellung}

\daniel{1.4 Team}

\samuel{1.5 webifier}

{2 Grundlagen}

\daniel{2.1 Frontend Technologien und Framework}

{2.2 Backend Technologien und Frameworks}

\indent \samuel{- Java}

\indent \samuel{- Spring}

\indent \samuel{- MongoDB}

\indent \jani{- Gradle}

\indent \jani{- Rest}

\indent \jani{- Docker}

\indent \jani{- R}

{2.3 Technologien und Frameworks der Tests}

\indent \daniel{- Python}

\indent \daniel{- PhantomJS}

\indent \jani{- Bro}

\indent \samuel{- HTtrack}

\indent \samuel{- Resemble.js}

{2.4 Angriffstypen}

\samuel{2.4.1 Malware}

\daniel{2.4.2 Request Header Investigation}

\jani{2.4.3 JavaScript Port \& IP Scanning}

\samuel{2.4.4 Phishing}

{3 Konzept}

{3.1 Gesamtkonzept}

\jani{3.1.1 webifier Tests}

\samuel{3.1.2 webifier Tester}

\samuel{3.1.3 webifier Platform}

\daniel{3.1.4 webifier Mail}

\samuel{3.1.5 webifier Data}

\jani{3.1.6 webifier Statistics}

{3.2 Testarten}

\samuel{3.2.1 Virenscan}

\daniel{3.2.2 Vergleich in verschiedenen Browsern}

\jani{3.2.3 Test auf Port Scanning}

\jani{3.2.4 Test auf IP Scanning}

\daniel{3.2.5 Link Checker}

\daniel{3.2.6 Google Safe Browsing}

\samuel{3.2.7 \"Uberpr\"ufung des Zertifikats}

\samuel{3.2.8 Erkennung von Phishing}

\jani{3.2.9 Screenshot}

{4 Umsetzung}

{4.1 Gesamtanwendung}

\jani{4.1.1 webifier Tests}

\samuel{4.1.2 webifier Tester}

\samuel{4.1.3 webifier Platform}

\daniel{4.1.4 webifier Mail}

\samuel{4.1.5 webifier Data}

\jani{4.1.6 webifier Statistics}

{4.2 Tests}

\samuel{3.2.1 Virenscan}

\daniel{3.2.2 Vergleich in verschiedenen Browsern}

\jani{4.2.3 Test auf Port Scanning}

\jani{4.2.4 Test auf IP Scanning}

\daniel{4.2.5 Link Checker}

\daniel{4.2.6 Google Safe Browsing}

\samuel{4.2.7 \"Uberpr\"ufung des Zertifikats}

\samuel{4.2.8 Erkennung von Phishing}

\jani{4.2.9 Screenshot}

{5 Analyse}

{6 Ausblick}

{6.1 Weitere Tests}

{6.2 Weitere Module}

{7 Fazit}

{7.1 Zusammenfassung}

{7.2 Bewertung der Ergebnisse}

\newpage

\section*{TEIL B: Vollständige Konfigurationsdatei webifier Tester}
\label{app:b}

\begin{scriptsize}
\lstset{
    style=eclipsejavascript,
    caption={Vollständige Konfigurationsdatei webifier Tester}
}
\begin{lstlisting}
{
  "resolver": {
    "name": "resolver",
    "startup": "docker run --rm --name #ID -e URL=#URL -e ID=#ID webifier-resolver",
    "startup_timeout_seconds": 60,
    "shutdown": "docker stop #ID",
    "shutdown_timeout_seconds": 30
  },
  "tests": [
    {
      "name": "VirusScan",
      "startup": "docker run --rm --name #ID -e URL=#URL -e ID=#ID webifier-test-virusscan",
      "startup_timeout_seconds": 600,
      "shutdown": "docker stop #ID",
      "shutdown_timeout_seconds": 30,
      "result_class": "de.securitysquad.webifier.output.result.virusscan.TestVirusScanResultInfo",
      "weight": 5,
      "enabled": true
    },
    {
      "name": "HeaderInspection",
      "startup": "docker run --rm --name #ID -e URL=#URL -e ID=#ID webifier-test-header-inspection",
      "startup_timeout_seconds": 300,
      "shutdown": "docker stop #ID",
      "shutdown_timeout_seconds": 30,
      "result_class": "de.securitysquad.webifier.output.result.headerinspection.HeaderInspectionResultInfo",
      "weight": 1,
      "enabled": true
    },
    {
      "name": "PortScan",
      "startup": "docker run --rm --name #ID -e URL=#URL -e ID=#ID webifier-test-portscan",
      "startup_timeout_seconds": 300,
      "shutdown": "docker stop #ID",
      "shutdown_timeout_seconds": 30,
      "result_class": "de.securitysquad.webifier.output.result.portscan.TestPortScanResultInfo",
      "weight": 3,
      "enabled": true
    },
    {
      "name": "IpScan",
      "startup": "docker run --rm --name #ID -e URL=#URL -e ID=#ID webifier-test-ipscan",
      "startup_timeout_seconds": 300,
      "shutdown": "docker stop #ID",
      "shutdown_timeout_seconds": 30,
      "result_class": "de.securitysquad.webifier.output.result.ipscan.TestIpScanResultInfo",
      "weight": 3,
      "enabled": true
    },
    {
      "name": "Screenshot",
      "startup": "docker run --rm --name #ID -e URL=#URL -e ID=#ID webifier-test-screenshot",
      "startup_timeout_seconds": 300,
      "shutdown": "docker stop #ID",
      "shutdown_timeout_seconds": 30,
      "result_class": "de.securitysquad.webifier.output.result.screenshot.TestScreenshotResultInfo",
      "weight": 0,
      "enabled": true
    },
    {
      "name": "LinkChecker",
      "startup": "docker run --rm --name #ID -e URL=#URL -e ID=#ID webifier-test-linkchecker",
      "startup_timeout_seconds": 300,
      "shutdown": "docker stop #ID",
      "shutdown_timeout_seconds": 30,
      "result_class": "de.securitysquad.webifier.output.result.linkchecker.TestLinkCheckerResultInfo",
      "weight": 1,
      "enabled": true
    },
    {
      "name": "CertificateChecker",
      "startup": "docker run --rm --name #ID -e URL=#URL -e ID=#ID webifier-test-certificatechecker",
      "startup_timeout_seconds": 300,
      "shutdown": "docker stop #ID",
      "shutdown_timeout_seconds": 30,
      "result_class": "de.securitysquad.webifier.output.result.certificatechecker.TestCertificateCheckerResultInfo",
      "weight": 3,
      "enabled": true
    },
    {
      "name": "PhishingDetector",
      "startup": "docker run --rm --name #ID -e URL=#URL -e ID=#ID webifier-test-phishingdetector",
      "startup_timeout_seconds": 300,
      "shutdown": "docker stop #ID",
      "shutdown_timeout_seconds": 30,
      "result_class": "de.securitysquad.webifier.output.result.phishingdetector.TestPhishingDetectorResultInfo",
      "weight": 5,
      "enabled": true
    },
    {
      "name": "GoogleSafeBrowsing",
      "startup": "docker run --rm --name #ID -e URL=#URL -e ID=#ID -e API_KEY=INSERT_API_KEY webifier-test-google-safe-browsing",
      "startup_timeout_seconds": 300,
      "shutdown": "docker stop #ID",
      "shutdown_timeout_seconds": 30,
      "result_class": "de.securitysquad.webifier.output.result.googlesafebrowsing.TestGoogleSafeBrowsingResultInfo",
      "weight": 3,
      "enabled": true
    }
  ],
  "preferences": {
    "push_result_data": true
  }
}
\end{lstlisting}
\end{scriptsize}

\newpage

\section*{TEIL C: Vollständige Ergebnisberechnung webifier Tester}
\label{app:c}

\begin{scriptsize}
\lstset{
    style=eclipsejava,
    caption={Vollständige Ergebnisberechnung webifier Tester}
}
\begin{lstlisting}
private WebifierOverallTestResult calculateOverallResult() {
    int weightSum = #tests#.stream().map(WebifierTest::getData).mapToInt(WebifierTestData::getWeight).sum();
    int mostWeighted = #tests#.stream().map(WebifierTest::getData).mapToInt(WebifierTestData::getWeight).max().orElse(weightSum / 2);
    double maliciousMin = (double) mostWeighted / (double) weightSum;
    double suspiciousMin = Math.pow(maliciousMin, 2);

    int undefinedTestSum = #tests#.stream().filter(test -> test.getResult().getResultType() == WebifierResultType.##UNDEFINED##)
            .map(WebifierTest::getData).mapToInt(WebifierTestData::getWeight).sum();
    double undefinedPercentage = (double) undefinedTestSum / (double) weightSum;
    if (undefinedPercentage > #MAX_UNDEFINED_TEST_PERCENTAGE#) {
        return new WebifierOverallTestResult(WebifierResultType.##UNDEFINED##);
    }
    double result = 0;
    for (WebifierTest<TestResult> test : tests) {
        double testWeight = (double) test.getData().getWeight() / (double) weightSum;
        result += getTestResultValue(test.getResult().getResultType(), testWeight) * testWeight;
    }
    if (result >= maliciousMin) {
        return new WebifierOverallTestResult(WebifierResultType.##MALICIOUS##, result);
    }
    if (result >= suspiciousMin) {
        return new WebifierOverallTestResult(WebifierResultType.##SUSPICIOUS##, result);
    }
    return new WebifierOverallTestResult(WebifierResultType.##CLEAN##, result);
}

private double getTestResultValue(WebifierResultType type, double testWeight) {
    if (type == WebifierResultType.##MALICIOUS##) {
        return 1;
    }
    if (type == WebifierResultType.##SUSPICIOUS##) {
        return testWeight;
    }
    return 0;
}
\end{lstlisting}
\end{scriptsize}