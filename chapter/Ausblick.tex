\chapter{Ausblick}

Mit webifier wurde eine modulare und leicht zu erweiternde Basis zur automatisierten Überprüfung von Webseiten geschaffen. Dadurch können in Zukunft noch weitere Projekte gestartet werden um den Umfang von webifier zu erweitern.

Der Umfang von webifier kann sowohl in Form von neuen Tests zur Überprüfung weiterer clientseitigen Angriffsszenarien, als auch in Form neuer Teilanwendungen von webifier, welche neue Funktionen für die Endnutzer bereitstellen, erweitert werden. So kann die Bedrohlichkeit von Webseiten noch besser klassifiziert werden und dem Nutzer können noch genauere Informationen zur Verfügung gestellt werden.

Ein solches Modul könnte beispielsweise ein Browsermodul sein, welches automatisch vor dem Abrufen einer Webseite webifier Data nach Einträgen zu dieser Seite durchsucht und den Nutzer bei Bedrohungen entsprechend warnt. Eine andere Möglichkeit wäre es die Analyse direkt über den Browser anzustoßen, was allerdings gravierende Auswirkungen auf das Nutzerverhalten hätte, da die Überprüfung einer Webseite etwa vier Minuten in Anspruch nimmt.

Des Weiteren können bereits implementierte Test noch erweitert und so das Gesamtergebnis der Überprüfungen optimiert werden, da viele Tests noch sehr oberflächlich sind und alleine Potential für mehrere Arbeiten liefern.

Abschließend lässt sich festhalten, das webifier Dank seiner Modularität ein großes Potential und viele Möglichkeiten zur Weiterentwicklung und Verwirklichung neuer Ideen geschaffen hat.