\chapter{Ausblick}

Mit \textit{webifier} wurde eine modulare und leicht zu erweiternde Basis zur automatisierten Überprüfung
von Webseiten geschaffen. Der Umfang von \textit{webifier} kann in Zukunft durch zusätzliche Projekte
erweitert werden.

Der Umfang von \textit{webifier} kann sowohl in Form von neuen Tests zur Überprüfung weiterer clientseitigen
Angriffsszenarien, als auch in Form neuer Teilanwendungen, welche neue Funktionen für
die Endnutzer bereitstellen, erweitert werden. So kann die Bedrohlichkeit von Webseiten noch besser
klassifiziert werden und dem Nutzer können noch genauere Informationen zur Verfügung gestellt
werden.

Ein solches Modul könnte beispielsweise ein Browserplugin sein, welches automatisch vor dem Abrufen
einer Webseite \textit{webifier Data} nach Einträgen zu dieser Seite durchsucht und den Nutzer bei
Bedrohungen entsprechend warnt. Eine andere Möglichkeit wäre es die Analyse direkt über den Browser
anzustoßen, was allerdings gravierende Auswirkungen auf das Nutzerverhalten hätte, da die
Überprüfung einer Webseite etwa vier Minuten in Anspruch nimmt.

Des Weiteren können bereits implementierte Test noch erweitert und so das Gesamtergebnis der
Überprüfungen optimiert werden, da viele Tests noch sehr oberflächlich sind und alleine Potential
für mehrere Arbeiten liefern.

Außerdem kann durch weitere Überprüfungen von Websseiten der Datenbestand erweitert werden, was neue
Auswertungsmöglichkeiten für \textit{webifier Statistics} schafft. Mit größeren Datenmengen können
zuverlässiger statistische Zusammenhänge analysiert werden. Durch weitere Diagramme können neue
Erkenntnisse gewonnen werden.

Dies ließe sich beispielsweise durch eine neue Webcrawler-Erweiterung für \textit{webifier} realisieren. Diese würde dann alle externen Links einer Webseite extrahieren und automatisiert der Warteschlange hinzufügen. So könnten systematisch Informationen über ganze Bereiche des Internets gewonnen und ausgewertet werden.

\textit{webifier Plattform} und \textit{webifier Mail} könnten noch über einen gemeinsamen Scheduler abstrahiert
werden. Dieser könnte auch eine horizontale Skalierbarkeit der Überprüfung ermöglichen, welche
momentan noch nicht implementiert ist.

Des weiteren bietet das Darknet ein großes Potential um weitere Analysen zu starten. Hierfür
müsste ein neues Modul entwickelt werden welches die Webseiten abruft, da diese nicht über eine
normale Verbindung erreicht werden können.

Dank seiner Modularität bietet \textit{webifier} optimale Möglichkeiten für eine Weiterentwicklung.