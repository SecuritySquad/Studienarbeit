\chapter{Einleitung}

\section{Einführung}

\section{Team}

\section{Aufgabenstellung}
Anbieter von zwielichtigen Web-Angeboten greifen ihre User mit diversen Client-seitigen Methoden an. Beispiele für solche Angriffe sind Malware Downloads, Phishing, JavaScript Intranet Angriffe, oder Browser Exploits.

Ziel der Arbeit ist eine systematische Untersuchung der Aktivitäten von semi-legalen Webseiten im \ac{WWW}. Das erwartete Ergebnis ist ein Prüfportal, auf dem jene Webseiten automatisiert analysiert werden und Ergebnisse präsentiert werden sollen.

Nach dem ersten Schaffen einer Übersicht von interessanten Zielen, wie z.B. One-Click-Hoster oder File-sharing Sites sollen ausgewählte Webseiten manuell untersucht werden. Außerdem sollen verschiedene Angriffsszenarien zur weiteren Prüfung ausgewählt werden. Der Untersuchungsprozes der Webseiten soll im Verlauf dieser Arbeit stückweise automatisiert und in den Rahmen einer Prüfanwendung gebracht werden.

Abschließend sollen eine Vielzahl von Webseiten mit der Anwendung getestet und die Ergebnisse ausgewertet und dokumentiert werden.

\section{webifier}
webifier ist eine Anwendung, um Webseiten auf deren Seriosität und mögliche clientseitige Angriffe auf den Nutzer hin zu untersuchen.

\begin{itemize}
  \item Konzeption einer Evaluierungsplatform, basierend beispielsweise auf einem automatisch angesteuerten Web Browser in einer virtuellen Maschine.
\end{itemize}
