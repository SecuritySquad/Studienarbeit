\chapter{Einleitung}

Anbieter von zwielichtigen Web-Angeboten greifen ihre User mit diversen Client-seitigen Methoden an. Beispiele für solche Angriffe sind Malware Downloads, Phishing, JavaScript Intranet Angriffe oder Browser Exploits.

\begin{center}
\textbf{\enquote{Jeder Fünfte Opfer von Internetkriminalität}}\footcite[Zitat][297]{cybercrime}
\end{center}

Dieses Zitat macht deutlich welche Bedrohung vom Internet und insbesondere von Webseiten ausgeht. Normale Nutzer sind heutzutage im World Wide Web ein gefragtes Angriffsziel für webbasierte Angriffe. Häufig wird hierfür der Nutzer auf maliziöse Webseiten gelockt. Diese Webseiten nutzen dann unter anderem Sicherheitslücken im Browser des Nutzers um Schadsoftware zu verbreiten oder den Anwender auszuspähen. Die nachfolgende Studienarbeit beschäftigt sich mit diesen Webseiten und analysiert deren Bedrohungspotenzial.

\section{Aufbau der Arbeit}

Die Studienarbeit ist in sieben Kapitel unterteilt. Das erste Kapitel ist die Einleitung.
Hier werden die Rahmenbedingungen für die Arbeit erläutert und es wird ein Einblick in die
Hintergründe gegeben. Das nächste Kapitel, Grundlagen, behandelt Tools, die maßgeblich zur
Entwicklung der Lösung verwendet werden. Weiterhin werden clientseitige Probleme bzw. Angriffspunkte
fachlich aufbereitet, die in der Lösung angesprochen werden. Im dritten Kapitel
werden die Ergebnisse unseres Entwurfsprozesses dargestellt und begründet. Dabei wird grundsätzlich
zwischen der Gesamtanwendung und den Tests unterschieden. Diese Aufteilung findet sich auch im
darauffolgenden Kapitel, der Umsetzung, wieder. Dort wird hingegen auf die konkrete
Implementierung des Konzepts eingegangen und bedeutende Anwendungslogik anhand von Codebeispielen
erklärt. In Kapitel fünf, der Analyse, werden die Erkenntnisse der Arbeit präsentiert und danach
differenziert beurteilt. Zum Abschluss der Arbeit werden in Ausblick und Fazit Ideen und
Verbesserungsvorschläge für mögliche Folgeprojekte vorgetragen und die Arbeit abschließend
bewertet.

\section{Aufgabenstellung}
Ziel der Arbeit ist eine systematische Untersuchung der Aktivitäten von semi-legalen Webseiten im
\ac{WWW}. Das erwartete Ergebnis ist ein Prüfportal, auf dem jene Webseiten automatisiert
analysiert und Ergebnisse präsentiert werden sollen.

Nach dem Erstellen einer Übersicht von interessanten Zielen, wie z.B. One-Click-Hoster oder
File-sharing Seiten sollen ausgewählte Webseiten manuell untersucht werden. Außerdem sollen
verschiedene Angriffsszenarien zur weiteren Prüfung ausgewählt werden. Der Untersuchungsprozess der
Webseiten soll im Verlauf dieser Arbeit stückweise automatisiert und in den Rahmen einer
Prüfanwendung gebracht werden.

Abschließend sollen eine Vielzahl von Webseiten mit der Anwendung getestet und die Ergebnisse
ausgewertet und dokumentiert werden.

\section{Team}
Das Entwicklerteam besteht aus drei Studenten der angewandten Informatik:
Samuel Philipp, Daniel Brown und Jan-Eric Gaidusch.
Der Name der Arbeitsgruppe ist \textit{SecuritySquad}.
\footnote{Der Name \textit{SecuritySquad} ist angelehnt an den Titel des US-amerikanischen Actionfilms \textit{Suicide Squad}.}

\begin{figure}[H]
	\centering
	\includegraphics[width=5cm]{images/securitysquad}
	\caption{SecuritySquad - Logo}
	\label{fig:securitysquad-logo}
\end{figure}

Die Studienarbeit wird von Dr. Martin Johns betreut, der an der DHBW Karlsruhe die Vorlesung Datensicherheit hält. Hauptberuflich ist er Forscher eben dieses Gebietes am CEC Karlsruhe der SAP SE.\footcite[Vgl.][]{johnsProfile}

\section{webifier}

\begin{figure}[H]
  \centering
  \includegraphics[width=5cm]{images/webifier}
  \caption{\textit{webifier} - Logo}
  \label{fig:webifier-logo}
\end{figure}

\textit{webifier} ist eine Anwendung mit der Webseiten auf deren Seriosität und mögliche
clientseitige Angriffe auf den Nutzer geprüft werden können. Sie besteht aus mehreren
eigenständigen Teilanwendungen. Im Zentrum steht der Tester, welcher die einzelnen Tests
verwaltet, ausführt und anschließend die Ergebnisse auswertet. Jeder einzelne Test ist eine weitere isolierte
Teilanwendung des Testers. So kann jeder Test unabhänig von allen anderen betrieben werden.

\textit{webifier Plattform} ist eine Webanwendung welche den Endnutzern eine grafische
Oberfläche zur Verfügung stellt, um Webseiten zu überprüfen. Im Hintergrund nutzt die Plattform
den Tester. \textit{webifier Mail} ist ein Dienst mit dem Links aus E-Mails überprüft werden können.
Anschließend erhält der Sender eine E-Mail mit den Resultaten zurück.

Eine weitere Teilanwendung ist \textit{webifier Data}. Sie stellt eine Schnittstelle
für den Tester bereit, um alle Testergebisse sammeln zu können. \textit{webifier Statistics} ist die
letzte Teilanwendung von \textit{webifier}. Sie nutzt die vom Data-Modul gespeicherten Daten um
Auswertungen aller Testergebnisse bereitzustellen.

Um das Konzept und die Umsetzung von \textit{webifier} verstehen zu können sind einige Grundlagen
erforderlich, welche nun im nächsten Kapitel genauer vorgestellt werden.
