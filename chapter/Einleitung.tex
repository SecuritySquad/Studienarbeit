\chapter{Einleitung}

\section{Einführung}

\todo Samuel

\section{Hintergrund}

\todo Jani

\section{Team}

\begin{figure}[H]
  \centering
  \includegraphics[width=5cm]{images/securitysquad}
  \caption{Secutitysquad - Logo}
  \label{fig:securitysquad-logo}
\end{figure}

\section{Aufgabenstellung}
Anbieter von zwielichtigen Web-Angeboten greifen ihre User mit diversen Client-seitigen Methoden an. Beispiele für solche Angriffe sind Malware Downloads, Phishing, JavaScript Intranet Angriffe, oder Browser Exploits.

Ziel der Arbeit ist eine systematische Untersuchung der Aktivitäten von semi-legalen Webseiten im \ac{WWW}. Das erwartete Ergebnis ist ein Prüfportal, auf dem jene Webseiten automatisiert analysiert werden und Ergebnisse präsentiert werden sollen.

Nach dem ersten Schaffen einer Übersicht von interessanten Zielen, wie z.B. One-Click-Hoster oder File-sharing Sites sollen ausgewählte Webseiten manuell untersucht werden. Außerdem sollen verschiedene Angriffsszenarien zur weiteren Prüfung ausgewählt werden. Der Untersuchungsprozes der Webseiten soll im Verlauf dieser Arbeit stückweise automatisiert und in den Rahmen einer Prüfanwendung gebracht werden.

Abschließend sollen eine Vielzahl von Webseiten mit der Anwendung getestet und die Ergebnisse ausgewertet und dokumentiert werden.

\section{webifier}

\begin{figure}[H]
  \centering
  \includegraphics[width=5cm]{images/webifier}
  \caption{webifier - Logo}
  \label{fig:webifier-logo}
\end{figure}

webifier ist eine Anwendung, mit der Webseiten auf deren Seriosität und mögliche clientseitige Angriffe auf den Nutzer geprüft werden können. Sie besteht aus mehreren eigenständigen Teilanwendungen. Im Zentrum steht der Tester, welcher die einzelnen Tests verwaltet, ausführt und anschließend die Ergebnisse auswertet. Jeder einzelne Test ist eine weitere isolierte Teilanwendung des Testers. So kann jeder Test unabhänig von allen anderen betrieben werden.

Die Platform ist eine Webanwendung welche den Endnutzern eine grafische Oberfläche zur Verfügung stellt, um Webseiten zu überprüfen. Im Hintergrund setzt die Plattform auf den Tester auf. webifier Mail ist ein Dienst mit dem Links aus E-Mails überprüft werden können. Anschließend erhält der Sender eine E-Mail mit den Resultaten zurück.

Eine weitere Teilanwendung von webifier ist das Data-Modul. Es stellt eine Schnittstelle für den Tester bereit, um alle Testergebisse sammeln zu können. Das Statisitik-Modul ist die letzte Teilanwendung von webifier. Es setzt auf das Data-Modul auf und stellt Funktionen zur Auswertung aller Testergebnisse bereit.

Um die Techniken und Algorithmen von webifier verstehen zu können sind einige Grundlagen erforderlich, welche nun im nächsten Kapitel genauer vorgsetellt werden.
