\chapter{Konzept}

In diesem Kapitel werden das Gesamtkonzept und die Konzepte der einzelnen Tests vorgestellt. Das Gesamtkonzept umfasst die Einzelnen Komponenten von webifier und deren Zusammenspiel. Im Folgenden wird nun das Gesamtkonzept beschrieben.

\section{Gesamtkonzept}

\todo{Daniel}

\subsection{webifier Tests}
Webifier Tests ist der Oberbegriff für sämtliche von webifier durchgeführten Tests bei der Analyse einer Webseite. Wie bei der gesamten Anwendung wird auch bei den Tests viel Wert auf Modularität gelegt. Jeder Test bildet ein eigenständiges Bauteil, welches nach belieben integriert oder entfernt werden kann ohne Effekte auf die Lauffähigkeit der Gesamtanwendung.

Da webifier auf die Analyse von maliziösen Seiten ausgelegt ist gibt es bei den Tests einige Punkte zu beachten um das System vor Viren und Schadcode zu schützen.
Jeder Test wird in einer vom Gesamtsystem abgekapselten Laufzeitumgebung ausgeführt. Aus einem Test heraus darf nicht auf das System zugegriffen werden, da die Tests gegebenenfalls mit Schadcode befallen werden können durch das Erforschen von maliziösen Seiten. Es soll vermieden werden, dass sich Schadcode oder Viren von den Tests auf den Server verbreiten. Nach Durchlauf und Übermittelung des Ergebnisses löscht der Test sich selbst und alle Laufzeitdaten. Als Ergebnis werden keinerlei Dateien versendet, es beschränkt sich auf eine Weitergabe des Ergebnisses in Form einer Zeichenkette. Damit soll vermieden werden, dass sich eventuell mit Viren befallene Dateien weiter auf dem System ausbreiten können.

Ein Test liefert sein Ergebnis an den Tester, welcher dies dann im folgenden weiterverarbeitet.
Das Starten und Organisisieren der Tests wird von webifier Tester durchgeführt. Den Aufbau und die Funktionsweise des Testers wird im nächsten Abschnitt beschrieben.

\subsection{webifier Tester}

Der webifier Tester verwaltet alle Tests, führt diese aus und berechnet aus den einzelnen Ergebnissen der Tests ein Gesamtergebnis.

\todo{Samuel}

\begin{itemize}
  \item Berechnung Gesamtergebnis
  \item Ergebnis => Data
\end{itemize}

\subsection{webifier Platform}

\todo{Samuel}

\begin{itemize}
  \item Userschnittstelle
  \item Warteschlange
\end{itemize}

\subsection{webifier Mail}

\todo{Daniel}

\subsection{webifier Data}

\todo{Samuel}

\begin{itemize}
  \item Notwendige Informationen
  \item Datenmodell
\end{itemize}

\subsection{webifier Statistics}
Webifier Statistics ist die Statistikoberfläche von webifier. Hier werden alle Daten der analysierten Webseiten aufbereitet und in visueller Form dargestellt. Die Daten stammen aus den Ergebnissen aller Tests, welche von webifier Data abgespeichert wurden.

Webifier Statistics liefert dem Nutzer eine Vielzahl an verschiedenen Graphen, welche bestimmte Teilaspekte beleuchten. Diese enthalten zum Einen die Gesamtauswertungen, welche sich mit der allgemeinen Datenauswertung jedes Gesamttests beschäftigen. Zum Anderen gibt es noch die Einzelauswertungen der Tests, die testspezifische Ergebnisse auswerten.

Alle Auswertungen werden dem Nutzer über eine Weboberfläche zugänglich gemacht. Als Einstieg gibt es ein \textit{Dashboard} mit einigen Zahlen und Fakten zu den Aktivitäten auf webifier. Auf die einzelnen Auswertungen wird in der Auswertung genauer eingegangen.

\section{Testarten}

\subsection{Virenscan}

\todo{Samuel}

\begin{itemize}
  \item Httrack (Umsetzung)
  \item Download aller Dateien der Webseite
  \item Scannen der Heruntergeladenen Dateien
  \begin{itemize}
    \item Clamav (Umsetzung)
    \item AVG (Umsetzung)
    \item CAV (Umsetzung)
  \end{itemize}
\end{itemize}

\subsection{Vergleich in verschiedenen Browsern}

\todo{Daniel}

\subsection{Test auf Port Scanning}
Der Test auf Port Scanning analysiert die Nutzung der Ports einer Webseite. Hierfür wird die Webseite automatisch vom Test geöffnet und dessen JavaScript ausgeführt. Parallel dazu muss die Netzwerkaktivität überwacht werden. Es werden alle eingehenden Anfragen auf das Testsystem zunächst geloggt. Da das Testsystem abgekapselt vom restlichen System ist, ist es irrelevant von welcher IP-Adresse die Anfragen kommen. Alle Anfragen lassen sich auf die aufgerufene Seite zurückführen, da restliche Netzwerkaktivität abgeschaltet ist. Dies ist wichtig, da es durchaus möglich ist das die Webseite nicht selbst einen Portscan-Angriff startet sondern beispielsweise über einen Drittrechner oder ein Botnetz gescannt wird. Zudem könnten die Ports auch lokal auf dem Client über JavaScript gescannt werden. Deshalb werden lediglich die angefragten Ports im Log gespeichert.

Nach erfolgreichem Durchschauen der Webseite beginnt die Analyse. Hier müssen alle Portanfragen klassifiziert werden. Es gibt eine Reihe von legitimen Portanfragen welche beispielsweise Port 80 für HTTP oder Port 443 für SSL sind. Diese Anfragen werden dann als harmlos markiert und somit ignoriert. Alle Anfragen, welche sich auf unspezifizierte Ports beziehen, werden als verdächtig markiert. Je nach Anzahl der verdächtigen Anfragen wird dann entschieden ob die Seite als bedrohlich, verdächtig oder sauber klassifiziert wird. Dieses Ergebnis wird dann mitsamt der gefundenen verdächtigen Portanfragen zurückgegeben.

\subsection{Test auf IP Scanning}
Der Test auf IP Scanning beschäftigt sich mit der Analyse der IP-Anfragen, welche durch eine Webseite ausgelöst werden. Wie auch bereits bei Portscanning beschrieben wird die Webseite automatisch geöffnet und dessen JavaScript ausgeführt. Die Netzwerküberwachung hat hier jedoch einen anderen Fokus. Es werden die IPs der gesendeten Anfragen geloggt. Beim IP Scanning wird grundsätzlich versucht über die bekannten Heimnetz-IP-Netze weitere im Netzwerk angeschlossene Geräte zu erkennen um beispielsweise Viren auf dem gesamten Netzwerk zu verbreiten. Diese Angriffe werden über JavaScript auf dem Clienten gestartet. Deshalb werden die vom Clienten gesendeten Anfragen protokolliert. Hiervon werden lediglich die IPs gespeichert.

Nach dem Speichern aller IPs werden diese klassifiziert. Der Test vergleicht alle Anfragen mit den bekannten Heimnetz-IPs (beispielsweise 192.168.178.*). Anfragen, welche sich nicht auf diese Adressen zurückführen lassen werden herausgefiltert, da diese irrelevant für den Test sind. Anhand der Anzahl der verdächtigen Adressen wird im Abschluss wieder die Seite klassifiziert und das Ergebnis mitsamt den Adressen zurückgeliefert an den Tester.

\subsection{Link Checker}

\todo{Daniel}

\begin{itemize}
  \item herausfiltern aller Links und nachgeladenen Ressourcen
\end{itemize}

\subsection{Google Safe Browsing}

\todo{Daniel}

\subsection{Überprüfung des Zertifikats}

\todo{Samuel}

\begin{itemize}
  \item Auslesen der relevanten Informationen des Zertifikates der WEbseite
  \item Validierung des Zertifikates
\end{itemize}

\subsection{Erkennung von Phishing}

\todo{Samuel}

\begin{itemize}
  \item Herausfiltern der Schlagwörter
  \item Finden möglicher Duplikate der Webseite
  \begin{itemize}
    \item Erstes Schlagwort zu Top Level Domains
    \begin{itemize}
      \item com
      \item ru
      \item net
      \item org
      \item de
    \end{itemize}
    \item Websuche nach den Schlagwörtern mittels Suchmaschinen
    \begin{itemize}
      \item DuckDuckGo
      \item Ixquick
      \item Bing
    \end{itemize}
  \end{itemize}
\end{itemize}

\subsection{Screenshot}
Der Screenshot-Test ist kein Test im eigentlichen Sinne. Er liefert keine Aussage über die Bedrohlichkeit einer Webseite. Trotzdem wurde er mit implementiert um den Nutzern einen Blick auf die Seite zu geben, welche sie von webifier haben scannen lassen. Dies kann besonders interessant sein, da viele Nutzer auch daran interessiert sind wie die Seiten denn aussehen und was dort an Text oder Bilder zu sehen ist. Jedoch sollte keiner der Nutzer, auf eine als bedrohlich markierte Webseite, mit seinem Webbrowser zugreifen. Deshalb wird hier die Möglichkeit gegeben sich gefahrlos einmal die Webseite anzuschauen.
