\chapter{Konzept}

In diesem Kapitel werden das Gesamtkonzept sowie die Konzepte der einzelnen Tests vorgestellt. Das
Gesamtkonzept umfasst die einzelnen Komponenten von \textit{webifier} und deren Zusammenspiel. Im
Folgenden wird zunächst das Gesamtkonzept beschrieben.

\section{Gesamtkonzept}

\textit{webifier} ist in Teilanwendungen gegliedert.
Dies hat den Zweck, bestehende Programme und zukünftige Erweiterungen getrennt
voneinander entwickeln zu können. In diesem Abschnitt wird die Entwurfsarchitektur der
Gesamtanwendung beschrieben. Dabei wird hauptsächlich auf den Zweck der Anwendungen und deren
Zusammenspiel eingegangen.

Grundsätzlich wird \textit{webifier} benutzt, um Webseiten anhand verschiedener Sicherheitsaspekte
zu untersuchen.
Abbildung \ref{fig:anwendung-konzept} zeigt eine Übersicht der Architektur und
Kommunikation des Systems.
Es wird der konzeptionelle Zusammenhang zwischen den Teilanwendungen erläutert, wobei die
Datenbank, da sie nicht eigenhändig entwickelt wurde, nicht näher beschrieben
wird.

Die Nutzerinteraktion erfolgt unter Anderem über \textit{webifier Plattform}. Diese stellt eine
Webseite bereit, auf welcher ein Anwender Seiten analysieren lassen kann. Die Plattform leitet die
Nutzeranfrage an \textit{webifier Tester} weiter.

Eine weitere Möglichkeit, \textit{webifier} zu verwenden, ist über \textit{webifier Mail}.
Hat der Nutzer eine verdächtige Spam-Mail erhalten, so kann er diese an ein, für \textit{webifier
Mail} bereitgestelltes Postfach weiterleiten.
Dort werden Links aus der Mail zusammengetragen und an den Tester weitergeleitet.

\begin{figure}[H]
	\centering
	\includegraphics[width=14cm]{images/anwendung-konzept}
	\caption{Architektur der Gesamtkonzepts}
	\label{fig:anwendung-konzept}
\end{figure}

Der Tester überprüft zunächst die \acs{URL} auf Korrektheit und löst ggf. Weiterleitungen
auf.
Entsteht dabei ein Fehler, wird der Testdurchlauf abgebrochen und der Fehler zurückgegeben.
Wurde die Auflösung der \acs{URL} erfolgreich ausgeführt, werden die Sicherheitstests des Testers
gestartet.

Im Tester wird die Ausführung der einzelnen Tests überwacht und koordiniert.
Wichtig dabei ist, dass alle Tests isoliert vom Rest des Systems laufen, da auf ihnen
schadhafte Webseiten aufgerufen werden können.
Jeder dieser Tests gibt ein eigenes Ergebnis, sowie relevante Zusatzinformationen zurück.
Diese Ergebnisse werden anschließend miteinander verrechnet und es wird eine Gesamteinschätzung mit
allen relevanten Informationen der einzelnen Tests erstellt.

Es wird zwischen den folgenden vier Ergebnistypen unterschieden: \textit{CLEAN},
\textit{SUSPICIOUS}, \textit{MALICIOUS} und \textit{UNDEFINED}.

\newpage

\begin{description}
    \item[unbedenklich] \hfill \texttt{CLEAN} \\
    Die Webseite verhält sich harmlos und ist ungefährlich im Bezug auf die getesteten Angriffe.
	\begin{figure}[H]
		\centering
		\includegraphics[scale=0.2]{images/webifier-clean}
		\caption{Icon für den Testergebnistyp \textit{CLEAN}}
	\end{figure}

	\item[verdächtig] \hfill \texttt{SUSPICIOUS} \\
	Die Webseite verhält sich bedenklich und ist suspekt.
	Diese Seite sollte gemieden werden.
    \begin{figure}[H]
    	\centering
    	\includegraphics[scale=0.2]{images/webifier-suspicious}
    	\caption{Icon für den Testergebnistyp \textit{SUSPICIOUS}}
    \end{figure}

    \item[bedrohlich] \hfill \texttt{MALICIOUS} \\
    Die Webseite verhält sich kritisch und ist gefährlich.
    Sie sollte unter keinen Umständen aufgerufen werden.
    \begin{figure}[H]
    	\centering
    	\includegraphics[scale=0.2]{images/webifier-malicious}
    	\caption{Icon für den Testergebnistyp \textit{MALICIOUS}}
    \end{figure}

	\item[unbekannt] \hfill \texttt{UNDEFINED} \\
	Der Test konnte aus technischen Gründen nicht abgeschlossen werden.
	Es kann deshalb keine Aussage über die Vertraueswürdigkeit der Seite gemacht werden.
	\begin{figure}[H]
		\centering
		\includegraphics[scale=0.2]{images/webifier-undefined}
		\caption{Icon für den Testergebnistyp \textit{UNDEFINED}}
	\end{figure}
\end{description}

Das berechnete Gesamtergebnis des Testers wird an \textit{webifier Plattform} bzw. \textit{webifier
Mail} weitergeleitet. Schließlich wird das Ergebnis dem Nutzer über den Browser bzw.
über eine Antwortmail präsentiert. Zusätzlich werden alle Ergebnisse des Testers an \textit{webifier
Data} geschickt und dort in einer Datenbank persistiert. Diese Daten werden von \textit{webifier
Statistics} für Analysen und Auswertungen genutzt

\subsection{webifier Tests}
\textit{webifier Tests} ist der Oberbegriff für sämtliche von \textit{webifier} durchgeführten Tests
bei der Analyse einer Webseite. Wie beim gesamten System wird auch bei den Tests viel Wert
auf Modularität gelegt. Jeder Test bildet ein eigenständiges Bauteil, welches nach Belieben
ohne Effekte auf die Lauffähigkeit des Gesamtsystems integriert oder entfernt werden kann.

Da \textit{webifier} auf die Analyse von maliziösen Seiten ausgelegt ist gibt es bei den Tests
einige Punkte zu beachten um das System vor Malware und Schadcode zu schützen.
Jeder Test wird in einer isolierten Laufzeitumgebung ausgeführt. Aus einem Test
heraus darf nicht auf das System zugegriffen werden, da die Tests, durch das Erforschen von
maliziösen Seiten, gegebenenfalls mit Schadcode befallen sein können. Es soll vermieden werden,
dass sich Schadcode oder Malware aus den Tests auf dem Server verbreiten. Nach Durchlauf und
Übermittelung des Ergebnisses löscht der Test sich selbst und alle zugrhörigen Daten. Die
Übermittelung beschränkt sich auf eine Weitergabe des Ergebnisses in Form einer Zeichenkette um zu
vermeiden, dass sich, eventuell mit Viren befallene, Dateien weiter auf dem System ausbreiten
können.

Ein Test liefert sein Ergebnis an den Tester, welcher dies im Folgenden weiterverarbeitet.

Das Starten und Organisisieren der Tests wird vom \textit{webifier Tester} durchgeführt, dessen
Aufbau und Funktionsweise im nächsten Abschnitt beschrieben wird.

\textit{webifier} stellt, zum Zeitpunkt dieser Arbeit, 9 verschiedene Tests um eine Webseite zu
klassifzieren.
Das Konzept der einzelnen Tests wird separat in Abschnitt
\ref{sec:konzept-testarten} erläutert.

In Tabelle \ref{tbl:tests} werden die Tests kurz beschrieben.

\begin{table}[H]
\centering
\begin{tabularx}{\textwidth}{|l|X|}
\hline
\textbf{Test} & \textbf{Beschreibung} \\\hline
\textit{Virenscan der Webseite} & Testet die Dateien einer Seite auf Malware \\\hline
\textit{Vergleich in verschiedenen Browsern} & Test ob sich die Seite auf verschiedenen Systemen
oder Browsern unterschiedlich verhält \\\hline
\textit{Überprüfung der Port-Nutzung} & Überprüft ob die Seite Port-Scanning betreibt \\\hline
\textit{Überprüfung der IP-Nutzung} & Überprüft ob die Seite IP-Scanning betreibt \\\hline
\textit{Prüfung aller verlinkten Seiten} & Sucht nach Einträgen zu den Links der Webseite
in \textit{webifier Data} \\\hline
\textit{Google Safe Browsing} & Nutzt eine Google-API um die Webseite und deren Links von Google
einstufen zu lassen \\\hline
\textit{Überprüfung des SSL-Zertifikats} & Überprüft das \acs{SSL}-Zertifikat der Webseite \\\hline
\textit{Erkennung von Phishing} & Testet ob es sich um eine Phishingseite handelt \\\hline
\textit{Screenshot der Seite} & Liefert dem Nutzer einen Screenshot der Webseite \\\hline
\end{tabularx}
\caption{Beschreibung der einzelnen Tests}
\label{tbl:tests}
\end{table}

\subsection{webifier Tester}
\label{sec:konzept-tester}

Der \textit{webifier Tester} verwaltet alle Tests, führt diese aus und berechnet aus den einzelnen
Ergebnissen der Tests ein Gesamtergebnis. Alle auszuführenden Tests werden in einer
Konfigurationsdatei angegeben und können deshalb dynamisch angepasst werden. Da jeder Test in einem
eigenen Prozess läuft wird beim Starten des Testers die Konfigurationsdatei geladen. Anschließend
werden die einzelnen Tests ausgeführt und auf ihr Ergebnis gewartet. Liegt von allen Tests das
Ergebnis vor wird ein Gesamtergebis berechnet. Die Berechnung dieses Ergebnisses wird im Folgenden
genauer erklärt.

Das Ergebnis kann entweder unbedenklich (\textit{CLEAN}), verdächtig (\textit{SUSPICIOUS}),
bedrohlich (\textit{MALICIOUS}) oder unbekannt (\textit{UNDEFINED}) sein. Für die Berechnung des
Gesamtergebnisses erhält jeder Test wie in Tabelle \ref{tbl:test-weights} dargestellt eine Gewichtung,
da manche Tests mehr über die Vertrauenswürdigkeit oder Gefahr einer Webseite aussagen als andere.
Am meisten fallen der \textit{Virenscan der Webseite} und die \textit{Erkennung von Phishing} ins
Gewicht, da dieses die ausschlaggebendsten Tests sind. Am wenigsten gewichtet sind der
\textit{Vergleich in verschiedenen Browsern}, weil dies nur ein Indiz ist, da es auch viele
Webseiten, wie die von YouTube, Nachrichtensendern, Blogs oder Ähnliche gibt, welche
dynamischen Inhalt bereitstellen und die \textit{Prüfung der verlinkten Seiten}, da dies immer vom
Datenbestand abhängt. Der \textit{Screenshot der Seite} fällt nicht ins Gewicht, da dies kein Test
im eigentlichen Sinne ist, sondern nur eine zusätzliche Information für den Nutzer darstellt. In
Abschnitt \ref{sec:konzept-testarten} wird die Wahl der Gewichtungen für die einzelnen Tests noch
ausführlicher erläutert.

\begin{table}[H]
\centering
\begin{tabular}{|l|l|l|l|}
\hline
\textbf{Test} & \textbf{Gewichtung} & \textbf{Prozentuale Gewichtung} \\\hline
\textit{Virenscan der Webseite} & 5 & \textasciitilde0,208\\\hline
\textit{Vergleich in verschiedenen Browsern} & 1 & \textasciitilde0,042\\\hline
\textit{Überprüfung der Port-Nutzung} & 3 & 0,125\\\hline
\textit{Überprüfung der IP-Nutzung} & 3 & 0,125\\\hline
\textit{Prüfung aller verlinkten Seiten} & 1 & \textasciitilde0,042\\\hline
\textit{Google Safe Browsing} & 3 & 0,125\\\hline
\textit{Überprüfung des SSL-Zertifikats} & 3 & 0,125\\\hline
\textit{Erkennung von Phishing} & 5 & \textasciitilde0,208\\\hline
\textit{Screenshot der Seite} & 0 & 0\\\hline
\end{tabular}
\caption{Gewichtungen der einzelnen Tests}
\label{tbl:test-weights}
\end{table}

Die prozentuale Gewichtung ergibt sich aus $\frac{Testgewichtung}{Summe~der~Gewichungen~aller~Tests}$.

Ein weiterer wichtiger Punkt, der für die Berechung des Gesamtergebnisses festgelegt wurde ist, dass mindestens 50\% aller Tests (berechnet anhand der prozentualen Gewichtung) ein bekanntes Ergebnis, also \textit{CLEAN}, \textit{SUSPICIOUS} oder \textit{MALICIOUS} haben müssen. Ist der Anteil bekannter Ergebnisse kleiner lässt sich kein zuverlässiges Ergebnis berechnen, da dieses sonst von zu wenigen ausschlaggebenden Faktoren abhängen würde.

Liefern also mehr als die Hälfte der Tests ein bekanntes Ergebnis, so kann daraus nun das Gesamtergebnis berechnet werden. Hierfür wird für jedes Testergebnis ein Wert zwischen 0 und 1 berechnet, welcher anschließend mit der prozentualen Gewichtung des Tests multipliziert wird. Die Werte der einzelnen Testergebnisse ergeben sich wie in Tabelle \ref{tbl:test-values} dargestellt. Ist das Testergebnis \textit{CLEAN} oder \textit{UNDEFINED} ist der Ergebniswert 0 und geht so nicht weiter in die Wertung ein. Ist das Ergebnis \textit{MALICIOUS} wird der Ergebniswert 1. Dadurch fällt das Gewicht dieses Tests voll in die Wertung. Ist das Ergebnis \textit{SUSPICIOUS} so wird die prozentuale Gewichtung des Tests als Ergebniswert gewählt. So fließt dieser Test mit dem Quadrat der Gewichtung in das Gesamtergebnis ein.

\begin{table}[H]
\centering
\begin{tabular}{|l|l|l|l|}
\hline
\textbf{Testergebnis} & \textbf{Ergebniswert}\\\hline
\textit{CLEAN} & 0\\\hline
\textit{SUSPICIOUS} & Prozentuale Gewichtung des Tests\\\hline
\textit{MALICIOUS} & 1\\\hline
\textit{UNDEFINED} & 0\\\hline
\end{tabular}
\caption{Zuordnung Testergebnis zu Ergebniswert}
\label{tbl:test-values}
\end{table}

Anschließend werden die Werte aller Tests zu einem Gesamtergebnis aufsummiert. Daraus ergibt sich ein Minimalwert von 0 und ein Maximalwert von 1. Dieser Wertebereich wird nun wie folgt auf die drei Ergebnisse \textit{CLEAN}, \textit{SUSPICIOUS} und \textit{MALICIOUS} verteilt. Die Tests mit der größten Gewichtung sollen hierbei ausschlaggebend sein. Daraus ergibt sich die prozentuale Gewichtung des Tests mit der größten Gewichtung als Minimalwert für \textit{MALICIOUS} und das Quadrat der prozentualen Gewichtung des Tests mit der größten Gewichtung als Minimalwert für \textit{SUSPICIOUS}.

Daraus lässt sich für die Werte der Tests aus Tabelle \ref{tbl:test-weights} die folgende Werteverteilung der Gesamtergebniswerte ableiten:

\begin{center}
$0 \leq CLEAN < 0,043402\overline{7} \leq SUSPICIOUS < 0,208\overline{3} \leq MALICIOUS \leq 1$
\end{center}

Zusätzlich zu dem berechneten Gesamtergebnis stellt der Tester auch alle Ergebnisse der Einzeltests
und deren spezifischen Testinformationen bereit. Außerdem werden alle Ergebnisse zu Persistierung an
das Modul \textit{webifier Data} gesendet, welches in Abschnitt \ref{sec:konzept-data} genauer erläutert wird.

\subsection{webifier Plattform}

\textit{webifier Plattform} ist eine Webanwendung, die den Tester nutzt und eine \ac{UI} für diesen
zur Verfügung stellt. Außerdem bereitet sie die Ergebnisse des Testers grafisch für den Benutzer auf.

Der Nutzer hat die Möglichkeit auf der ersten Seite von \textit{webifier Plattform} eine \acs{URL}
einzugeben, welche getestet werden soll. Da die Kapazität jedes Systems beschränkt ist verwaltet die
Plattform alle Anfragen zur Webseitenüberprüfung in einer Warteschlange. In einer
Konfigurationsdatei kann angegeben werden, wie viele Tests parallel ausgeführt werden sollen. So
wird die Warteschlange nach und nach abgearbeit. Anschließend werden die Ergebnisse des Testers für
den Benutzer visuell aufbereitet. Das bereitgestellte Ergebnis umfasst zum Einen das Gesamtresultat, welches vom Tester berechnet wurde und zum Anderen sowohl die Ergebnisse der einzelnen Tests, als auch die zusätzlichen Informationen, welche von diesen bereitgestellt wurden.

So erhält der Nutzer einen umfassenden Bericht über die Vertrauenswürdigkeit oder die ausgehende Gefahr der überprüften Webseite.

\subsection{webifier Mail}
Mit dieser Komponente können verdächtige Mails über \textit{webifier Tester} überprüft werden.
Nutzer müssen dazu lediglich ihre empfangene Nachricht an ein für \textit{webifier Mail}
bereitgestelltes Postfach weiterleiten. Eine Beispielhafte E-Mail wird im Screenshot in
Abbildung \ref{fig:spam-mail} dargestellt. In \textit{webifier Mail} werden bis zu fünf in der
Mail enthaltene Links an den Tester weitergeleitet. Ist die Prüfung vollendet wird eine Antwortmail
mit den Ergebnissen an den Nutzer zurückgeschickt.

\begin{figure}[H]
	\centering
	\includegraphics[width=12cm]{images/spam-mail.png}
	\caption{Screenshot einer Spam-Mail}
	\label{fig:spam-mail}
\end{figure}

\subsection{webifier Data}
\label{sec:konzept-data}

\textit{webifier Data} ist die Persistenzkomponente von \textit{webifier}. \textit{webifier Tester}
nutzt \textit{webifier Data} um alle Testergebnisse an einem zentralen Ort abzulegen, egal wo dieser
ausgeführt wird.

Ein Testergebnis, welches in dem Datamodul gespeichert wird enthält einmal die eingegebene \acs{URL}
und die getestete \acs{URL}, das Gesamtergebnis, sowohl den Typ (\textit{CLEAN},
\textit{SUSPICIOUS}, \textit{MALICIOUS} oder \textit{UNDEFINED}), als auch den Ergebniswert. Außerdem wird die Testlaufzeit gespeichert. Zusätzlich werden noch weitere Informationen zu den einzelnen Tests gespeichert. Dazu zählen der Name, die Konfigurationsparameter, wie beispielsweise die Gewichtung, das Resultat und die Detailinformationen zu diesem.

Die Komponente stellt auch eine Schnittstelle zum Abfragen der bereits gespeicherten Ergebnisse
bereit. Diese wird beispielsweise vom Test zur \textit{Prüfung aller verlinkten Seiten} verwendet.
Die Funktionsweise dieses Tests wird in Abschnitt \ref{sec:konzept-linkchecker} erklärt.

Alle Ergebnisse werden in einer Datenbank abgelegt. Da die zusätzlichen Informationen der einzelnen
Tests teilweise sehr unterschiedlich sind, kommt hierfür keine relationale Datenbank in Frage.
\textit{webifier Data} nutzt deshalb zur Speicherung aller Daten die dokumentbasierte Datenbank
\textit{MongoDB}. Alle weiteren Informationen hierzu folgen im Umsetzungsteil dieser Arbeit in
Abschnitt \ref{sec:umsetzung-data}.

\subsection{webifier Statistics}
\textit{webifier Statistics} ist die Statistikoberfläche von \textit{webifier}. Hier werden alle
Daten der analysierten Webseiten aufbereitet und in visueller Form dargestellt. Die Daten stammen
aus den Ergebnissen aller Tests, welche von \textit{webifier Data} abgespeichert wurden.

\textit{webifier Statistics} liefert dem Nutzer eine Vielzahl an verschiedenen Graphen, welche
bestimmte Teilaspekte beleuchten. Diese enthalten zum Einen die Gesamtauswertungen, welche sich mit der allgemeinen Datenauswertung jedes Gesamttests beschäftigen. Zum Anderen gibt es noch die Einzelauswertungen der Tests, die testspezifische Ergebnisse auswerten.

Alle Auswertungen werden dem Nutzer über eine Weboberfläche zugänglich gemacht. Als Einstieg gibt es
ein Dashboard mit einigen Zahlen und Fakten zu den Aktivitäten auf \textit{webifier}. Auf
die einzelnen Auswertungen wird in Kapitel \ref{cha:analyse} genauer eingegangen.

\section{Testarten}
\label{sec:konzept-testarten}

In diesem Abschnitt werden nun die einzelnen Tests vorgestellt, mit welchen die zu überprüfenden
Webseiten analysiert werden. Wie bereits erwähnt werden alle dieser Tests vom Tester verwaltet und
ausgeführt.

\subsection{Virenscan der Webseite}

Der \textit{Virenscan der Webseite} nutzt verschiedene Antivirenprogramme um die Webseite auf
Malware zu überprüfen.
Um dies zu realisieren wird zunächst die Webseite inklusive aller enthaltenen Dateien und Links
heruntergeladen und gespeichert. Anschließend werden die Virenscanner gestartet, welche die
heruntergeladenen Dateien überprüfen. Abschließend werden alle Ergebnisse der einzelnen Scans
zusammengeführt und daraus ein Endergebnis berechnet.

Für das Endergebnis werden zunächst alle gescannten Dateien klassifiziert. Wird eine Datei von
keinem der Virenscanner als Malware eingestuft wird diese als \textit{CLEAN} gekennzeichnet.
Hält nur ein Virenscanner die Datei für Malware, wird diese als \textit{SUSPICIOUS}
eingestuft. Stufen mehr als ein Virenscanner eine Datei als Malware ein, so ist diese
\textit{MALICIOUS}.
Sind alle Dateien \textit{CLEAN}, so ist auch das Endergeblis dieses Tests
\textit{CLEAN}. Sollten eine oder mehrere Dateien \textit{SUSPICIOUS} sein wird auch das Endergebnis
\textit{SUSPICIOUS}. Das selbe gilt danach für \textit{MALICIOUS}. Sobald eine Datei
\textit{MALICIOUS} ist, ist das Endergbnis ebenfalls \textit{MALICIOUS}.

Zusätzlich zum Endergebnis wird noch die gesamte Liste der gescannten Dateien inklusive der
jeweiligen Klassifizierung bereitgestellt und an den Tester weitergegeben.

\subsection{Vergleich in verschiedenen Browsern}

In Abschnitt \ref{sec:user-agent-sniffing} wurde bereits das Thema \textit{User Agent Sniffing}
behandelt.
Dieser Test soll genau dies aufgreifen.
Es sollen \acs{HTTP}-Anfragen von verschiedenen Browsertypen mit entsprechenden User Agents an den
Webserver der Zielseite geschickt werden.
Danach werden die Antworten untereinander auf Unterschiede untersucht. Dabei entstehen Kennwerte
für die Übereinstimmung der Antworten, die miteinander verrechnet werden müssen.
Dieser berechnete Wert für die Durchschnittsübereinstimmung muss schließlich anhand von passenden
Schwellwerten in die richtige Ergebnisklasse eingeteilt werden.

\subsection{Überprüfung der Port-Nutzung}
Die \textit{Überprüfung der Port-Nutzung} analysiert die Nutzung der Ports einer Webseite. Hierfür
wird die Webseite automatisch vom Test geöffnet und dessen JavaScript ausgeführt. Parallel dazu muss die
Netzwerkaktivität überwacht werden. Es werden alle eingehenden Anfragen auf das Testsystem zunächst
protokolliert. Da das Testsystem abgekapselt vom restlichen System ist, ist es irrelevant von
welcher IP-Adresse die Anfragen kommen. Alle Anfragen lassen sich auf die aufgerufene Seite
zurückführen, da restliche Netzwerkaktivität abgeschaltet ist. Dies ist wichtig, da es durchaus
möglich ist das die Webseite nicht selbst einen Portscan-Angriff startet sondern beispielsweise
über einen Drittrechner oder ein Botnetz gescannt wird. Zudem könnten die Ports auch lokal auf dem
Client über JavaScript gescannt werden. Deshalb werden lediglich die angefragten Ports im Log
gespeichert.

Nach erfolgreichem Durchsuchen der Webseite beginnt die Analyse. Hier müssen alle Portanfragen
klassifiziert werden. Es gibt eine Reihe von legitimen Portanfragen welche beispielsweise Port 80
für \acs{HTTP} oder Port 443 für \acs{SSL} sind. Diese Anfragen werden dann als harmlos markiert und
somit ignoriert. Alle Anfragen, welche sich auf unspezifizierte Ports beziehen, werden als
verdächtig markiert. Je nach Anzahl der verdächtigen Anfragen wird dann entschieden ob die Seite
als \textit{bedrohlich}, \textit{verdächtig} oder \textit{unbedenklich} klassifiziert wird. Dieses
Ergebnis wird dann mitsamt der gefundenen verdächtigen Portanfragen zurückgegeben.

\subsection{Überprüfung der IP-Nutzung}
Die \textit{Überprüfung der IP-Nutzung} beschäftigt sich mit der Analyse der IP-Anfragen, welche
durch eine Webseite ausgelöst werden. Wie auch bereits bei der \textit{Überprüfung der Port-Nutzung}
beschrieben wird die Webseite automatisch geöffnet und dessen JavaScript ausgeführt. Die
Netzwerküberwachung hat hier jedoch einen anderen Fokus. Es werden die IPs der gesendeten Anfragen
protokolliert. Beim IP Scanning wird grundsätzlich versucht über die bekannten Heimnetz-IP-Netze
weitere im Netzwerk angeschlossene Geräte zu erkennen um beispielsweise Viren im gesamten Netzwerk
zu verbreiten. Diese Angriffe werden über JavaScript auf dem Client gestartet. Deshalb werden die
vom Browser gesendeten Anfragen protokolliert. Hiervon werden lediglich die IPs gespeichert.

Nach dem Speichern aller IPs werden diese klassifiziert. Der Test vergleicht alle Anfragen mit den
bekannten Heimnetz-IPs (beispielsweise 192.168.178.*). Anfragen, welche sich nicht auf diese
Adressen zurückführen lassen werden herausgefiltert, da diese irrelevant für den Test sind. Anhand
der Anzahl der verdächtigen Adressen wird im Abschluss wie bei der \textit{Überprüfung der
Port-Nutzung} die Seite klassifiziert und das Ergebnis mitsamt den Adressen an den
Tester zurückgeliefert.

\subsection{Prüfung aller verlinkten Seiten}
\label{sec:konzept-linkchecker}

Dieser Test sucht innerhalb des Quellcodes der Anwendung nach Hyperlinks und referenzierten Ressourcen.
Für alle gefundenen Hosts wird dann geprüft, ob \textit{webifier Data} bereits Einträge dazu
bereitstellt.
Gibt es einen Eintrag, der \textit{SUSPICIOUS} ist, dann ist das ganze Endergebnis \textit{SUSPICIOUS}.
Sobald ein Eintrag \textit{MALICIOUS} ist, so ist auch das Gesamtergebnis \textit{MALICIOUS}.
Ansonsten liefert der Test \textit{CLEAN} zurück. Falls jedoch mehr als 34\% der Hosts
\textit{UNDEFINED} sind, dann ist auch das Gesamtergebnis \textit{UNDEFINED}.

\subsection{Google Safe Browsing}

\begin{center}
	\enquote{Safe Browsing is a Google service that lets client applications check \acsp{URL} against
	Google's constantly updated lists of unsafe web resources.}\footcite[Vgl.][]{googleSafeBrowsing}
\end{center}

Dieser Cloud-Service erlaubt es der Anwendung also eine weitaus größere Datenbank als ihre Eigene zu
nutzen.

Der Service unterscheidet zwischen vier Bededrohungstypen\label{par:konzep-gsb-types}: Malwareseiten, Sozialmanipulationsseiten, ungewollte Software und potenziell schadhafte Webanwendungen.
Weitere Vorteile sind die Seriösität des Unternehmens und die einfach zu bewerkstelligende
Integration des Services.
Leider können die Einträge aus der Datenbank des Services nicht exportiert werden, sondern nur gegen Listen von Links gecheckt werden.
In diesem Test werden wie bei der \textit{Prüfung der verlinkten Seiten} alle Links zu Webseiten und
Resourcen zusammengetragen.
Diese werden dann aber nicht an \textit{webifier Data}, sondern an den Google Safe Browsing Dienst
weitergeleitet.
Sobald ein Treffer aus dem Service zurückgemeldet wird, ist das Ergebnis \textit{SUSPICIOUS}.
Falls sich mehr als 40\% der Links in der Datenbank befinden, so wird die Webseite als
\textit{MALICIOUS} eingestuft.
Wird die zu testende Webseite selbst von Google Safe Browsing erkannt, so ist das Ergebnis ebenfalls
\textit{MALICIOUS}.
Wird kein Eintrag gefunden, so gilt die Seite als \textit{CLEAN}.

\subsection{Überprüfung des SSL-Zertifikats}

Die \textit{Überprüfung des SSL-Zertifikats} der Webseite sucht nach einem vorhandenen Zertifikat
und validiert dieses, sofern die Webseite eines nutzt. Hierfür liest es die dafür notwendigen
Informationen des Zertifikats aus und berechnet anschließend ein Testergebnis.

Stellt die Webseite kein Zertifikat zur Verfügung so ist das Testergebnis \textit{SUSPICIOUS}, da
es in Zeiten von Let's Encrypt\footnote{\url{https://letsencrypt.org/}} jedem möglich ist ein
\acs{SSL}-Zertifikat kostenlos zu erwerben und so die Sicherheit der eigenen Webseite zu erhöhen.
Nutzt die Webseite ein valides Zertifikat ist das Ergebnis \textit{CLEAN}. Weist das Zertifikat
Fehler auf ist das Ergebnis \textit{MALICIOUS}. Solche Fehler können beispielsweise sein, dass das
Zertifikat abgelaufen ist, dass es selbst signiert wurde oder das es für den falschen Host genutzt
wird.

\subsection{Erkennung von Phishing}

\textit{webifier} enthält auch einen sehr einfachen Test zur \textit{Erkennung von Phishing}. Dieser
sucht zuerst nach den Schlagwörtern der gegebenen Webseite. Hierfür zählt er die Häufigkeit aller
vorkommenden Wörter die mehr als drei Buchstaben haben. Wörter in Bildbeschreibungen und
Überschriften werden doppelt gewichtet, Wörter im Titel der Webseite werden fünffach gewichtet.
Haben mehrere Wörter die gleiche Gewichtung, so fällt die Länge der Wörter mit ins Gewicht und
längere Wörter werden bevorzugt. Die vier Wörter, die im Ranking am höchsten stehen werden
anschließend als Schüsselwörter gewählt.

Folgend werden mit Hilfe öffentlicher Suchmaschinen mögliche Duplikate der Webseite gesucht.
Für diese Suche werden die ausgewählten Schlagwörter verwendet. Nun werden die Ergebnisse aller
Suchmaschinen zusammengeführt und ebenfalls gewichtet. Je mehr Suchmaschinen einen Link gefunden
haben, desto höher steigt dieser Link im Ranking. Als nächstes muss diese Liste der möglichen
Duplikate nach Orginalen, welcher der gegebenen Webseite entsprechen gefilterert werden, da es sehr
wahrscheinlich ist, dass diese ebenfalls in der Liste der Links enthalten ist. Als letztes wird die
Liste noch auf maximal zehn Eintrage gekürzt.

Nun werden alle gefundenen Links mit der gegebenen Webseite verglichen und für jeden gefundenen Link
wird ein Ergebnis berechnet. Der Vergleich erfolgt auf drei Ebenen: es werden der Inhalt und der
Quelltext der beiden Webseiten, aber auch das Aussehen verglichen. Jeder dieser Vergleiche gibt die
prozentuale Übereinstimmung der zu vergleichenden Webseiten zurück. Anschließend werden diese drei
Ergebnisse zu einem Gesamtergebnis aggregiert. In diese Rechnung fließt das Ergebnis des
Screenshotvergleichs mit doppeltem Gewicht ein, da dieser Vergleich am aussagekräftigsten ist.

Aufgrund der Komplexität wird die genaue Berechnung des Ergebnisses in Abschnitt \ref{sec:umsetzung-phishungdetector} erklärt und hier nun vereinfacht dargestellt. Stimmen die beiden Webseiten zu 80\% überein, so ist das Ergebnis des Links \textit{SUSPICIOUS}, stimmen die beiden Webseiten zu 90\% überein, so ist das Ergebnis \textit{MALICIOUS}. Das Endergebnis des Tests wird abschließend wie folgt berechnet: wurde mindestens ein verdächtiger Link gefunden, so ist das Gesamtergebnis \textit{SUSPICIOUS}, wurde mindestens ein bedrohlicher Link gefunden, so ist das Ergebnis \textit{MALICIOUS}, andernfalls \textit{CLEAN}. Zusätzlich werden noch die gefundenen Schlagwörter, die gefundenen Phishingseiten, deren Vergleichswerte und ein Überlagerungsscreenshot mit der Originalseite für den Tester bereitgestellt.

\subsection{Screenshot der Seite}
Der \textit{Screenshot der Seite} ist kein Test im eigentlichen Sinne. Er liefert keine Aussage über
die Bedrohlichkeit einer Webseite. Deshalb liefert er immer als Ergebnis \textit{unbedenklich} und
bleibt ungewichtet bei der Ergebnisberechnung im Tester. Er wurde als Zusatzinformation zu den
anderen Tests implementiert um den Nutzern einen Blick auf die Seite zu geben, welche von
\textit{webifier} überprüft wurde.
Dies kann besonders interessant sein, da viele Nutzer auch daran interessiert sind wie die Seiten
aussehen und was dort an Text oder Bildern zu sehen ist. Jedoch sollte keiner der Nutzer, auf
eine als bedrohlich markierte Webseite, mit seinem Webbrowser zugreifen. Deshalb wird ihm hier die
Möglichkeit gegeben sich die Webseite gefahrlos anzuschauen.
