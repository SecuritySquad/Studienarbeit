\chapter{Konzept}

In diesem Kapitel werden das Gesamtkonzept und die Konzepte der einzelnen Tests vorgestellt. Das Gesamtkonzept umfasst die Einzelnen Komponenten von webifier und deren Zusammenspiel. Im Folgenden wird nun das Gesamtkonzept beschrieben.

\section{Gesamtkonzept}

\todo{Daniel}

\subsection{webifier Tests}
Webifier Tests ist der Oberbegriff für sämtliche von webifier durchgeführten Tests bei der Analyse einer Webseite. Wie bei der gesamten Anwendung wird auch bei den Tests viel Wert auf Modularität gelegt. Jeder Test bildet ein eigenständiges Bauteil, welches nach belieben integriert oder entfernt werden kann ohne Effekte auf die Lauffähigkeit der Gesamtanwendung.

Da webifier auf die Analyse von maliziösen Seiten ausgelegt ist gibt es bei den Tests einige Punkte zu beachten um das System vor Viren und Schadcode zu schützen.
Jeder Test wird in einer vom Gesamtsystem abgekapselten Laufzeitumgebung ausgeführt. Aus einem Test heraus darf nicht auf das System zugegriffen werden, da die Tests gegebenenfalls mit Schadcode befallen werden können durch das Erforschen von maliziösen Seiten. Es soll vermieden werden, dass sich Schadcode oder Viren von den Tests auf den Server verbreiten. Nach Durchlauf und Übermittelung des Ergebnisses löscht der Test sich selbst und alle Laufzeitdaten. Als Ergebnis werden keinerlei Dateien versendet, es beschränkt sich auf eine Weitergabe des Ergebnisses in Form einer Zeichenkette. Damit soll vermieden werden, dass sich eventuell mit Viren befallene Dateien weiter auf dem System ausbreiten können.

Ein Test liefert sein Ergebnis an den Tester, welcher dies dann im folgenden weiterverarbeitet.
Das Starten und Organisisieren der Tests wird von webifier Tester durchgeführt. Den Aufbau und die Funktionsweise des Testers wird im nächsten Abschnitt beschrieben.

\subsection{webifier Tester}

Der webifier Tester verwaltet alle Tests, führt diese aus und berechnet aus den einzelnen Ergebnissen der Tests ein Gesamtergebnis.

\todo{Samuel}

\begin{itemize}
  \item Berechnung Gesamtergebnis
  \item Ergebnis => Data
\end{itemize}

\subsection{webifier Platform}

\todo{Samuel}

\begin{itemize}
  \item Userschnittstelle
  \item Warteschlange
\end{itemize}

\subsection{webifier Mail}

\todo{Daniel}

\subsection{webifier Data}

\todo{Samuel}

\begin{itemize}
  \item Notwendige Informationen
  \item Datenmodell
\end{itemize}

\subsection{webifier Statistics}
Webifier Statistics ist die Statistikoberfläche von webifier. Hier werden alle Daten der analysierten Webseiten aufbereitet und in visueller Form dargestellt. Die Daten stammen aus den Ergebnissen aller Tests, welche von webifier Data abgespeichert wurden.

Webifier Statistics liefert dem Nutzer eine Vielzahl an verschiedenen Graphen, welche bestimmte Teilaspekte beleuchten. Diese enthalten zum Einen die Gesamtauswertungen, welche sich mit der allgemeinen Datenauswertung jedes Gesamttests beschäftigen. Zum Anderen gibt es noch die Einzelauswertungen der Tests, die testspezifische Ergebnisse auswerten.

Alle Auswertungen werden dem Nutzer über eine Weboberfläche zugänglich gemacht. Als Einstieg gibt es ein \textit{Dashboard} mit einigen Zahlen und Fakten zu den Aktivitäten auf webifier. Auf die einzelnen Auswertungen wird in der Auswertung genauer eingegangen.

\section{Testarten}

\subsection{Virenscan}

\todo{Samuel}

\begin{itemize}
  \item Httrack (Umsetzung)
  \item Download aller Dateien der Webseite
  \item Scannen der Heruntergeladenen Dateien
  \begin{itemize}
    \item Clamav (Umsetzung)
    \item AVG (Umsetzung)
    \item CAV (Umsetzung)
  \end{itemize}
\end{itemize}

\subsection{Vergleich in verschiedenen Browsern}

\todo{Daniel}

\subsection{Test auf Port Scanning}

\todo{Jani}

\subsection{Test auf IP Scanning}

\todo{Jani}

\subsection{Link Checker}

\todo{Daniel}

\begin{itemize}
  \item herausfiltern aller Links und nachgeladenen Ressourcen
\end{itemize}

\subsection{Google Safe Browsing}

\todo{Daniel}

\subsection{Überprüfung des Zertifikats}

\todo{Samuel}

\begin{itemize}
  \item Auslesen der relevanten Informationen des Zertifikates der WEbseite
  \item Validierung des Zertifikates
\end{itemize}

\subsection{Erkennung von Phishing}

\todo{Samuel}

\begin{itemize}
  \item Herausfiltern der Schlagwörter
  \item Finden möglicher Duplikate der Webseite
  \begin{itemize}
    \item Erstes Schlagwort zu Top Level Domains
    \begin{itemize}
      \item com
      \item ru
      \item net
      \item org
      \item de
    \end{itemize}
    \item Websuche nach den Schlagwörtern mittels Suchmaschinen
    \begin{itemize}
      \item DuckDuckGo
      \item Ixquick
      \item Bing
    \end{itemize}
  \end{itemize}
\end{itemize}

\subsection{Screenshot}

\todo{Jani}
