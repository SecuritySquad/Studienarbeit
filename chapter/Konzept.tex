\chapter{Konzept}

\section{Gesamtkonzept}

\subsection{webifier Tests}

\todo Jani

\subsection{webifier Tester}

\todo Samuel

\subsection{webifier Platform}

\todo{Daniel}

\subsection{webifier Mail}

\todo{Daniel}

\subsection{webifier Data}

\todo Samuel

\subsection{webifier Statistics}

\todo Jani
Webifier Statistics ist für die Datenauswertung zuständig. Das Ziel hierbei ist es die Daten zu visualisieren. Es sollen Diagramme erstellt und in einem Dashboard zusammen dargestellt werden. Die Datengrundlage wird von webifier Data geliefert. Alle Testergebnisse und Metadaten müssen analysiert werden. Durch diese Analyse werden Ideen gesammelt, welche Diagramme erstellt werden sollen.

Diese Diagramme werden folgend erstellt und getestet ob sinnvolle Erkenntnisse daraus gezogen werden können. Die fertigen Diagramme werden anschließend in das Dashboard eingebunden. Webifier Statistics soll für den Nutzer ein interaktives Dashboard bieten, auf dem er Einblicke über die von webifier gewonnen Daten erhalten kann. 

\section{Testarten}

\subsection{Virenscan}

\todo Samuel

\begin{itemize}
  \item Httrack (Umsetzung)
  \item Download aller Dateien der Webseite
  \item Scannen der Heruntergeladenen Dateien
  \begin{itemize}
    \item Clamav (Umsetzung)
    \item AVG (Umsetzung)
    \item CAV (Umsetzung)
  \end{itemize}
\end{itemize}

\subsection{Vergleich in verschiedenen Browsern}

\todo{Daniel}

\subsection{Test auf Port Scanning}

\todo Jani

\subsection{Test auf IP Scanning}

\todo Jani

\subsection{Link Checker}

\todo{Daniel}

\begin{itemize}
  \item herausfiltern aller Links und nachgeladenen Ressourcen
\end{itemize}

\subsection{Google Safe Browsing}

\todo{Daniel}

\subsection{Überprüfung des Zertifikats}

\todo Samuel

\begin{itemize}
  \item Auslesen der relevanten Informationen des Zertifikates der WEbseite
  \item Validierung des Zertifikates
\end{itemize}

\subsection{Erkennung von Phishing}

\todo Samuel

\begin{itemize}
  \item Herausfiltern der Schlagwörter
  \item Finden möglicher Duplikate der Webseite
  \begin{itemize}
    \item Erstes Schlagwort zu Top Level Domains
    \begin{itemize}
      \item com
      \item ru
      \item net
      \item org
      \item de
    \end{itemize}
    \item Websuche nach den Schlagwörtern mittels Suchmaschinen
    \begin{itemize}
      \item DuckDuckGo
      \item Ixquick
      \item Bing
    \end{itemize}
  \end{itemize}
\end{itemize}

\subsection{Screenshot}

\todo Jani
