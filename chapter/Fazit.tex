\chapter{Fazit}

Die Motivation dieser Arbeit war es den Nutzern des Internets eine Möglichkeit zu bieten Webseiten
auf ihre Vertrauenswürdigkeit, bzw. ihre Bedrohlichkeit hin überprüfen zu lassen. Hierfür sollten
die Aktivitäten von Webseiten anhand verschiedener Tests analysiert und ausgewertet werden. Mit
webifier wurde ein solches System geschaffen.

Bereits in der Konzeption wurden Modularität und einfache Erweiterbarkeit des Systems als Ziele
verfolgt. So wurde webifier in seine Teilanwendungen gegliedert, welche über einheitliche
Schnittstellen miteinander kommununizieren. Dem Nutzer wurde die Nutzung über webifier Plattform und
webifier Mail ermöglicht. 

Aus diesem Kozept folgte die Entwicklung der einzelnen Teilanwendungen. Diese wurden mit weit
verbreiteten Technolorien und Frameworks umgesetzt. Außerdem war es notwendig die Tests in einer
isolierten Umgebung auszuführen, um das eigene System vor Angriffen und möglichem Schaden zu
schützen.

Darauf folgend wurde eine großflächige Analyse von Webseiten durchgeführt, deren Ergebnisse in
webifier Data gespeichert wurden. Anschließend nutzte webifier Statistics diese Daten um
Auswertungen zu erstellen und diese für den Nutzer zu visualisieren. Bei diesen Diagrammen wurden
die gesammelten Daten genau betrachtet. In der Folge wurden die gewonnenen Erkenntnisse
untersucht und bewertet.

Abschließend lässt sich festhalten, das webifier ein großes Potential und viele Möglichkeiten zur
Weiterentwicklung und Verwirklichung neuer Ideen geschaffen hat.