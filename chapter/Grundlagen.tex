\chapter{Grundlagen}

In diesem Kapitel werden die Grundlagen, welche für das weitere Verständnis der Arbeit und der gesamten Anwendung notwendig sind, näher beschrieben. Zunächst werden die verschiedenen Technologien und Frameworks, sowohl des Frontends, als auch des Backends dargestellt. Anschließend werden einige gängige Angriffstypen im \ac{WWW} erläutert, welche webifier überprüft.

\section{Frontend Technologien und Frameworks}

\todo{Daniel}
\begin{itemize}
    \item HTML
    \item CSS
    \item JavaScript
    \item jQuery
    \item Bootstrap
\end{itemize}

\section{Backend Technologien und Frameworks}

In diesem Abschnitt werden nun alle Technologien und Frameworks vorgestellt welche in den Backends der einzelnen Teilanwendungen zum Einsatz kamen.

Wohl am häufigsten kam die Programmiersprache Java zum Einsatz. Java ist eine universal einsetzbare, nebenläufige, klassenbarierte und objektorientierte Programmiersprache. Sie wurde möglichst einfach gestaltet um von vielen Entwicklern genutzt zu werden. In ihrer Syntax ähnelt sie den Programmiersprachen C und C++. Außerdem ist sie stark und statisch typisiert. Vorallem aber zeichnet sich Java durch seine plattformunabhängigkeit aus. Diese wird dadurch umgesetzt, dass Java-Quellcode in plattformunabhängigen Byte-Code kompiliert wird, welcher von einer \ac{JVM} ausgeführt wird. Java ist eine Hochsprache, die mit Hilfe des so genannten \enquote{Garbage Collectors} eine automatische Speicherverwaltung bereitstellt. \footcite[Vgl.][1]{javaspecification}

In einigen Teilprojekten wurde das auf Java basierende \textit{Spring}-Framework verwendet. \textit{Spring} stellt eine vereinfachte Möglichkeit auf den Zugriff auf viele \ac{API} der Standard-Version zur Verfügung. Ein weiterer wesentlicher Bestandteil des \textit{Spring}-Frameworks ist die \textit{Dependency Injection}. Hierbei suchen sich Objekte ihre Referenzen nicht selbst, sondern bekommen diese Anhand einer Konfiguration injiziert. Dadurch sind sie eigenständig und können in verschiedenen Umgebungen eingesetzt werden. Des weiteren bringt \textit{Spring} eine Unterstützung für aspektorientierte Programmierung mit, wodurch mit verschiedenen Abstraktionsschichten einzelne Module abgekapselt werden können. \footcite[Vgl.][2]{spring3}

Aufbauend auf dem \textit{Spring} Basis-Modul werden noch weitere Module, wie beispielsweise Spring Security, Sprint Boot, Spring Integration, Spring Data, Spring Session oder Spring MVC. \footcite[Vgl.][2]{springPivotal} Im folgenden werden die \textit{Spring}-Mudule naäher erläutert, die für das weitere Verständnis der Arbeit notwendig sind.

\begin{description}
  \item[Spring Boot] \hfill \\
    Mit Spring Boot können Anwendungen, welche das \textit{Spring}-Framework nutzen, einfacher eintwickelt und ausgeführt werden, da dadurch eigenständig lauffähige Programme erzeugt werden können, welche nicht von externen Services abhängig sind. Hierfür bringt Spring Boot einen Integrierten Server mit, auf welchem die Anwendung bereitgestellt wird.\footcite[Vgl.][1]{springBoot}
  \item[Spring MVC] \hfill \\
    Spring MVC ist sehr gut geeignet um Webanwendungen zu implementieren.\footcite[Vgl.][3]{spring3} Hierfür können die diese in mehrere Abstraktionsschichten gegliedert werden. Beispielsweise in das \ac{UI}, die Geschäftslogik und die Persistenzschicht.\footcite[Vgl.][21]{springMvc}
  \item[Spring Data] \hfill \\
    Spring Data ist\ldots
\end{description}


Ein wichtiger Bestandteil jedes großen Software-Projektes ist ein gutes Build-Management-Tool. Für webifier wurde \textit{Gradle} als solches gewählt. Ein Build-Prozess besteht grundsätzlich aus zwei Teilschritten. Zum Einen aus dem kompilieren des Codes und zum anderen aus dem verlinkten der benutzen Bibliotheken. \cite{buildprozess}
Da das manuelle Einbinden von Bibliotheken und compilieren des Codes bei großen Projekten sehr aufwändig und mühsam sein kann wird hier auf Build-Management-Tools wie \textit{Gradle} zurückgegriffen. Um den Build für den Nutzer möglichst einfach zu gestalten verfolgt Gradle zwei Prinzipien. Das erste Prinzip ist \textit{Convention over Configuration}, was bedeutet, dass soweit es geht ein Standardbuildprozess definiert ist und der Anwender nur die Parameter ändern muss die Projektspezifisch abweichen. Das zweite Prinzip nennt sich \acl{DRY}. Hierbei geht es darum Redundanzen in der Konfiguration des Buildes zu vermeiden. Diese beiden Prinzipien helfen Gradle, dass meist kurze Build-Skripte ausreichen um komplexe Prozesse abzubilden. \footcite[Vgl.][6f]{gradle}

Die Kommunikation zwischen Server und Client erfolgt über \acl{REST}. Jedes Objekt wird in \ac{REST} als Ressource definiert, welche über einen eindeutigen \acl{URI} adressiert werden können. Über die HTTP-Methoden GET,PUT,POST und DELETE können diese Ressourcen geladen, erstellt, geändert oder auch gelöscht werden. \cite{rest}



\begin{itemize}
  \item MongoDB\newline \todo Samuel
  \item Docker  \newline \todo Jani
  \item R \newline \todo Jani
\end{itemize}

\section{Technologien und Frameworks der Tests}

\begin{itemize}
    \item Python \newline \todo{Daniel}
    \item Node JS
    \item Phantom JS \newline \todo{Daniel}
    \item Bro \newline \todo Jani
    \item HTtrack\newline \todo Samuel
    \item Resemble JS\newline \todo Samuel
\end{itemize}


\section{Angriffstypen}

\subsection{Malware}

\todo Samuel

\subsection{Request Header Investigation}

\todo{Daniel}

\subsection{JavaScript Port Scanning}

\todo Jani

\subsection{JavaScript IP Scanning}

\todo Jani

\subsection{Clickjacking}

\todo Jani

\subsection{Phishing}

\todo Samuel
