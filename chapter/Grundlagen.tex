\chapter{Grundlagen}

In diesem Kapitel werden die Grundlagen, welche für das weitere Verständnis der Arbeit und der gesamten Anwendung notwendig sind, näher beschrieben. Zunächst werden die verschiedenen Technologien und Frameworks, sowohl des Frontends, als auch des Backends dargestellt. Anschließend werden einige gängige Angriffstypen im \ac{WWW} erläutert, welche webifier überprüft.

\section{Frontend Technologien und Frameworks}

\todo{Daniel}
\begin{itemize}
    \item HTML
    \item CSS
    \item JavaScript
    \item jQuery
    \item Bootstrap
\end{itemize}

\section{Backend Technologien und Frameworks}

In diesem Abschnitt werden nun alle Technologien und Frameworks vorgestellt welche in den Backends der einzelnen Teilanwendungen zum Einsatz kamen.

Wohl am häufigsten kam die Programmiersprache Java zum Einsatz. Java ist eine universal einsetzbare, nebenläufige, klassenbarierte und objektorientierte Programmiersprache. Sie wurde möglichst einfach gestaltet um von vielen Entwicklern genutzt zu werden. In ihrer Syntax ähnelt sie den Programmiersprachen C und C++. Außerdem ist sie stark und statisch typisiert. Vorallem aber zeichnet sich Java durch seine plattformunabhängigkeit aus. Diese wird dadurch umgesetzt, dass Java-Quellcode in plattformunabhängigen Byte-Code kompiliert wird, welcher von einer \ac{JVM} ausgeführt wird. Java ist eine Hochsprache, die mit Hilfe des so genannten \enquote{Garbage Collectors} eine automatische Speicherverwaltung bereitstellt. \footcite[Vgl.][1]{javaspecification}

In einigen Teilprojekten wurde das auf Java basierende \textit{Spring}-Framework verwendet. \textit{Spring} stellt eine vereinfachte Möglichkeit auf den Zugriff auf viele \ac{API} der Standard-Version zur Verfügung. Ein weiterer wesentlicher Bestandteil des \textit{Spring}-Frameworks ist die \textit{Dependency Injection}. Hierbei suchen sich Objekte ihre Referenzen nicht selbst, sondern bekommen diese Anhand einer Konfiguration injiziert. Dadurch sind sie eigenständig und können in verschiedenen Umgebungen eingesetzt werden. Des weiteren bringt \textit{Spring} eine Unterstützung für aspektorientierte Programmierung mit, wodurch mit verschiedenen Abstraktionsschichten einzelne Module abgekapselt werden können. \footcite[Vgl.][2]{spring3}

Aufbauend auf dem \textit{Spring} Basis-Modul werden noch weitere Module, wie beispielsweise Spring Security, Sprint Boot, Spring Integration, Spring Data, Spring Session oder Sprint Web Services. \footcite[Vgl.][2]{springPivotal} \ldots

\todo Samuel

\begin{itemize}
  \item MongoDB\newline \todo Samuel
  \item Gradle  \newline \todo Jani
  \item REST  \newline \todo Jani
  \item Docker  \newline \todo Jani
  \item R \newline \todo Jani
\end{itemize}

\section{Technologien und Frameworks der Tests}

\begin{itemize}
    \item Phantom JS \newline \todo{Daniel}
    \item Bro \newline \todo Jani
    \item Python \newline \todo{Daniel}
    \item HTtrack\newline \todo Samuel
    \item Resemble JS\newline \todo Samuel
\end{itemize}


\section{Angriffstypen}

\subsection{Malware}

\todo Samuel

\subsection{Request Header Investigation}

\todo{Daniel}

\subsection{JavaScript Port Scanning}

\todo Jani

\subsection{JavaScript IP Scanning}

\todo Jani

\subsection{Clickjacking}

\todo Jani

\subsection{Phishing}

\todo Samuel
