\chapter{Grundlagen}

In diesem Kapitel werden die Grundlagen, welche für das weitere Verständnis der Arbeit und der gesamten Anwendung notwendig sind, näher beschrieben. Zunächst werden die verschiedenen Technologien und Frameworks, sowohl des Frontends, als auch des Backends dargestellt. Anschließend werden einige gängige Angriffstypen im \ac{WWW} erläutert, welche webifier überprüft.

\section{Frontend Technologien und Frameworks}

\begin{itemize}
    \item HTML
    \item CSS
    \item JavaScript
    \item jQuery
    \item Bootstrap
\end{itemize}

\section{Backend Technologien und Frameworks}

In diesem Abschnitt werden nun alle Technologien und Frameworks vorgestellt welche in den Backends der einzelnen Teilanwendungen zum Einsatz kamen.

Wohl am häufigsten kam die Programmiersprache Java zum Einsatz. Java ist eine universal einsetzbare, nebenläufige, klassenbarierte und objektorientierte Programmiersprache. Sie wurde möglichst einfach gestaltet um von vielen Entwicklern genutzt zu werden. In ihrer Syntax ähnelt sie den Programmiersprachen C und C++. Außerdem ist sie stark und statisch typisiert. Vorallem aber zeichnet sich Java durch seine plattformunabhängigkeit aus. Diese wird dadurch umgesetzt, dass Java-Quellcode in plattformunabhängigen Byte-Code kompiliert wird, welcher von einer \ac{JVM} ausgeführt wird. Java ist eine Hochsprache, die mit Hilfe des so genannten \enquote{Garbage Collectors} eine automatische Speicherverwaltung bereitstellt. \footcite[Vgl.][1]{javaspecification}

In einigen Teilprojekten wurde das auf Java basierende \textit{Spring}-Framework verwendet.

\begin{itemize}
  \item Spring
  \item MongoDB
  \item REST
  \item Docker
  \item R
\end{itemize}

\section{Technologien und Frameworks der Tests}

\begin{itemize}
    \item Phantom JS
    \item Bro
    \item Python
    \item HTtrack
    \item Resemble JS
\end{itemize}


\section{Angriffstypen}

\subsection{Malware}

\subsection{Request Header Investigation}

\subsection{JavaScript Port Scanning}

\subsection{JavaScript IP Scanning}

\subsection{Clickjacking}

\subsection{Phishing}
